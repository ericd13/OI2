%-------------------------------------------------------------------------------
%-------------------------------------------------------------------------------
\chapter{Tas persistants}
%-------------------------------------------------------------------------------
%-------------------------------------------------------------------------------
\thispagestyle{empty}
%-------------------------------------------------------------------------------
%-------------------------------------------------------------------------------
\begin{abstract}
Dans le cours nous avons défini le type de tas qui a été implémenté à l'aide d'un tableau. On a ainsi utilisé une structure de données itératives, c'est-à-dire non persistante.

Nous allons dans ce travail revenir au type récursif d'arbre binaire.
\end{abstract}
%-------------------------------------------------------------------------------
%-------------------------------------------------------------------------------
%-------------------------------------------------------------------------------
\section{Présentation}
%-------------------------------------------------------------------------------
%-------------------------------------------------------------------------------
%-------------------------------------------------------------------------------
On veut implémenter le type de données abstraite de file de priorité. Dans ce TP, la priorité maximale correspondra à une clé minimale. On manipule des ensembles de couples {\it clé$\times$valeur} où la clé est un entier et la valeur est de type $\alpha$ (\type{'a}).

Dans ce TP nous simplifierons l'étude en ne manipulant que les clés entières.


On a donc besoin d'un type \type{fdp} avec les fonctions
\begin{enumerate}
\item \type{fpVide : int -> fdp}, 
\item \type{estVide : fdp -> bool},
\item l'ajout est \type{ajouter : fdp -> int -> fdp}, 

 on crée un nouvel objet
\item \type{premier : fdp ->  int} ,

permet de voir l'élément de priorité maximale (de clé minimale) et sa priorité,
\item le retrait de l'élément prioritaire est \type{enlever : fdp -> fdp}.
\end{enumerate}
%
\bigskip

Le type sera classiquement 
\begin{lstlisting}
type fdp = Vide | Noeud of fdp * int * fdp;;
\end{lstlisting}

La taille d'un arbre $a$ est notée $|a|$.

Notre structure de file de priorité sera implémentée par des tas de Braun croissants ; ce sont des arbres binaires qui vérifient les deux propriétés suivantes.
%-------------------------------------------------------------------------------
\begin{description}
  \item[croissance] : l'étiquette de tout nœud est inférieure aux étiquettes de ses fils,
  \item[équilibre fort] : pour tout nœud, \type{Noeud(g,x,d)}, $|d| \le |g| \le |d|+1$.
\end{description}
%-------------------------------------------------------------------------------
Ces propriétés devront être maintenues dans les constructions.
%-------------------------------------------------------------------------------
%-------------------------------------------------------------------------------
%-------------------------------------------------------------------------------
\section{Fonctions}
%-------------------------------------------------------------------------------
%-------------------------------------------------------------------------------
%-------------------------------------------------------------------------------
\begin{Exercise}[title = Test]\it 
Écrire une fonction \type{estBraun : fdp -> bool} qui renvoie true ou \type{false} selon que l'arbre passé en paramètre est ou non un tas de Braun.

On pourra écrire une fonction auxiliaire qui renvoie ce booléen, la taille de l'arbre et sa racine.
\end{Exercise}
%-------------------------------------------------------------------------------
\begin{Answer}
\begin{lstlisting}
let racine arbre = 
   match arbre with
   |Vide -> failwith "L'arbre est vide"
   |Noeud(_,x,_) -> x;;
  
let estBraun arbre = 
   let rec aux arbre =
      match arbre with
      |Vide -> true, 0, max_int
      |Noeud(Vide,x,Vide) -> true, 1, x
      |Noeud(g,x,Vide) -> let b, n, y = aux g in (n=1), (n+1), x
      |Noeud(Vide, x, d) -> let b, n, y = aux d in false, (n+1), x
      |Noeud(g, x, d) -> let bg, ng, xg = aux g in
                         let bd, nd, xd = aux d in
                         (xg >= x) && (xd >= x) && bg && bd && (nd <= ng) && (nd <= ng+1), 
                         ng + nd + 1, 
                         x
  in let b, n, x = aux arbre in b;;
\end{lstlisting}
\end{Answer}
%-------------------------------------------------------------------------------
%-------------------------------------------------------------------------------
\begin{Exercise}[title = Premières fonctions]\it 

Écrire les fonctions \type{fpVide}, \type{estVide} et \type{premier}.
\end{Exercise}
%-------------------------------------------------------------------------------
\begin{Answer}
\begin{lstlisting}
let fpVide n = Vide;;

let estVide tas = 
   tas = Vide;;

let premier tas =
  match tas with
  |Vide -> failwith "Le tas est vide"
  |Noeud(_, n, _) -> n;;
\end{lstlisting}
\end{Answer}
%-------------------------------------------------------------------------------
%-------------------------------------------------------------------------------
\subsection*{Ajout}
%-------------------------------------------------------------------------------
On remarque que, pour garder la condition d'équilibre fort lors de l'ajout, on ne peut ajouter un élément que sur le fils de droite mais qu'alors celui-ci devient le fils de plus grande taille et il faut inverser les fils.
%-------------------------------------------------------------------------------
%-------------------------------------------------------------------------------
\begin{Exercise}[title = Ajouter]\it 
Écrire une fonction \type{ajouter}.
\end{Exercise}
%-------------------------------------------------------------------------------
\begin{Answer}
\begin{lstlisting}
let rec ajouter x t =
   match t with
   |Vide -> Noeud(Vide, x, Vide)
   |Noeud(g,y,d) ->  if x > y
                     then Noeud(ajouter x d, y, g)
                     else Noeud(ajouter y d, x, g);;
\end{lstlisting}
\end{Answer}
%-------------------------------------------------------------------------------
%-------------------------------------------------------------------------------
\subsection*{Retrait}
%-------------------------------------------------------------------------------
Pour pouvoir enlever l'élément prioritaire on aura besoin de reconstruire un tas de Braun à partir de deux fils et d'un entier. On suppose, dans la question suivante que \type{a1} et \type{a2} sont deux tas de braun tels que $|a_2| \le |a_1| \le |a_2|+1$.
%-------------------------------------------------------------------------------
%-------------------------------------------------------------------------------
\begin{Exercise}[title = Enlever un élément]\it 
Écrire une fonction \type{extractionGauche : arbre -> int * arbre} qui reçoit un tas de Braun et qui renvoie l'élément le plus à gauche et un tas de Braun contenant les éléments restants.
\end{Exercise}
%-------------------------------------------------------------------------------
\begin{Answer}
\begin{lstlisting}
let rec extractionGauche arbre = 
   match arbre with
   |Vide -> failwith "L'arbre est vide"
   |Noeud(Vide, x,_) -> x, Vide
   |Noeud(g, x, d) -> let y, g1 = extractionGauche g in y, Noeud(d, x, g1);;
\end{lstlisting}
\end{Answer}
%-------------------------------------------------------------------------------
%-------------------------------------------------------------------------------
\begin{Exercise}[title = Reconstitution]\it 
Écrire une fonction \type{union : int -> arbre -> arbre -> arbre} qui reçoit un entier $k$ et deux tas de Braun de taille adéquate et qui renvoie un tas de Braun contenant les étiquettes des deux arbres et $k$ en plus.
\end{Exercise}
%-------------------------------------------------------------------------------
\begin{Answer} On commence par les cas où la racine finale n'est pas $k$, les autres cas se construisent simplement.
\begin{lstlisting}
let rec union k a1 a2 =
  match a1, a2 with
  |Noeud(g1, r1, d1), Noeud(g2, r2, d2) when (r1 < k) && (r1 < r2) 
        -> Noeud(union k g1 d1, r1, a2)
  |Noeud(g1, r1, d1), Noeud(g2, r2, d2) when (r2 < k) && (r2 <= r1) 
        -> Noeud(a1, r2, union k g2 d2)
  |Noeud(Vide, r, Vide), Vide when (r < k)
        -> Noeud(Noeud(Vide, k, Vide), r, Vide)
  |_, _ -> Noeud(a1, k, a2);;
\end{lstlisting}
\end{Answer}
%-------------------------------------------------------------------------------
%-------------------------------------------------------------------------------
\begin{Exercise}[title = Enlever la racine]\it 
En déduire une fonction \type{enlever : arbre -> arbre} qui renvoie un tas de Braun obtenu en enlevant la racine du tas passé en paramètre.
\end{Exercise}
%-------------------------------------------------------------------------------
\begin{Answer}
\begin{lstlisting}
let enlever arbre = 
  match arbre with
  |Vide -> failwith "L'arbre est vide"
  |Noeud(Vide, _, _) -> Vide
  |Noeud(g, x, d) -> let y, g1 = extractionGauche g in union y d g1;;
\end{lstlisting}
\end{Answer}
%-------------------------------------------------------------------------------
%-------------------------------------------------------------------------------
\begin{Exercise}[title = Question facultative]\it 
Déterminer une autre manière d'enlever l'élément prioritaire en remontant la racine d'un fils.
\end{Exercise}
%-------------------------------------------------------------------------------
\begin{Answer} On peut enlever récursivement la racine (\type{rg}) du fils gauche en modifiant celui-ci en \type{g1}. 

On note \type{d = Noeud(gd, rd, dd)} le fils droit. 
Si \type{rg <= rd} on renvoie \type{Noeud(d, rg, g1)} sinon on renvoie \type{Noeud(union rg gd dd, rd, g1)}.
\begin{lstlisting}
let rec enlever1 arbre = 
   match arbre with
   |Vide -> failwith "L'arbre est vide"
   |Noeud(g, r, Vide) -> g
   |Noeud(g, r, (Noeud(gd, rd, dd) as d))
           -> let rg = premier g in
              let rd = premier d in                                     
              let g1 = enlever1 g in 
              if rg < rd then Noeud(d, rg, g1)
                         else Noeud(union rg gd dd, rd, g1);;
\end{lstlisting}
\end{Answer}
%-------------------------------------------------------------------------------
%-------------------------------------------------------------------------------
\section{Calcul des nombres de {\sc Hamming}}
%-------------------------------------------------------------------------------
%-------------------------------------------------------------------------------
%-------------------------------------------------------------------------------
Les nombres de {\sc Hamming} sont les entiers qui n'admettent que 2, 3 et 5 comme diviseurs premiers, ce sont les entiers de l forme $n = 2^a.3^b.5^c$.

On remarque que si $x$ est un nombre de {\sc Hamming} alors $2x,3x,5x$ le sont aussi, et que tous les nombres de {\sc Hamming} à part $1$ s'obtiennent en multipliant par $2$ ou $3$ ou $5$ un nombre de {\sc Hamming} plus petit.

Pour obtenir les nombres de Hamming successifs, on peut
%-------------------------------------------------------------------------------
\begin{enumerate}
\item Initialiser une file de priorité avec le nombre 1, 
\item répéter $n$ fois :
extraire le premier élément de la file, $x$ et insérer $2x$, $3x$ et $5x$ dans la file.
\end{enumerate}
%-------------------------------------------------------------------------------
%-------------------------------------------------------------------------------
\begin{Exercise}[title = Une première tentative]\it 
Programmer cet algorithme.
\end{Exercise}
%-------------------------------------------------------------------------------
\begin{Answer}
\begin{lstlisting}
let rec hamming n =
   let rec aux n tas =
      match n with
      |1 -> premier tas
      |n -> let p = premier tas in
            let t1 = enlever tas in
            let t2 = ajouter (5*p) t1 in
            let t3 = ajouter (3*p) t2 in
            let t4 = ajouter (2*p) t3 in
            aux (n-1) t4
  in let t0 = ajouter 1 (fpVide 0) in aux n t0;;
\end{lstlisting}
\end{Answer}
%-------------------------------------------------------------------------------
%-------------------------------------------------------------------------------
\bigskip

Les 20 premiers nombres de Hamming sont 

1, 2, 3, 4, 5, 6, 8, 9, 10, 12, 15, 16, 18, 20, 24, 25, 27, 30, 32, 36.

L'algorithme précédent va produire 

1, 2, 3, 4, 5, 6, 6, 8, 9, 10, 10, 12, 12, 15, 15, 16, 18, 18, 18.

Le nombre 900 apparaît 90 fois ! En effet les éléments sont insérés plusieurs fois.
%-------------------------------------------------------------------------------
%-------------------------------------------------------------------------------
\begin{Exercise}[title = Correction]\it 
Écrire une fonction qui renvoie vraiment le $n$-ième nombre de Hamming.

On pourra vérifier : le 1000-ième nombre de Hamming est $51\,200\,000$.
\end{Exercise}
%-------------------------------------------------------------------------------
\begin{Answer}$x$ est un nombre de {\sc Hamming} extrait de la file.

Si $x$ est divisible par $5$, $x=5y$, alors $2x = 5.2y$ a été inséré auparavant car $2y < x$.

De même $3x = 5.3y$ est aussi présent. Il est donc inutile d'insérer ces nombres, seul $5x$ est nouveau. 

Lorsque $x$ n'est pas divisible par $5$ mais l'est par $3$ alors un raisonnement analogue montrer qu'il est inutile d'insérer $2x$.

\begin{lstlisting}
let rec hamming n =
   let rec aux n tas =
      match n with
      |1 -> premier tas
      |n -> let p = premier tas in
            let t1 = enlever tas in
            let t2 = ajouter (5*p) t1 in
            if p mod 5 = 0
            then aux (n-1) t2
            else let t3 = ajouter (3*p) t2 in
                 if p mod 3 = 0
                 then aux (n-1) t3
                 else let t4 = ajouter (2*p) t3 in
                      aux (n-1) t4
  in let t0 = ajouter 1 (fpVide 0) in aux n t0;;
\end{lstlisting}
\end{Answer}
%-------------------------------------------------------------------------------
%-------------------------------------------------------------------------------
%-------------------------------------------------------------------------------
\section{Nombres taupins} 
%-------------------------------------------------------------------------------
%-------------------------------------------------------------------------------
%-------------------------------------------------------------------------------
Dans les temps anciens, chaque taupin avait un numéro secret, son {\bf nombre taupin}.

Les premiers étudiants avaient le numéro 1.

Pour les étudiants suivant le nombre était attribué par le parrain.
\begin{itemize}
    \item Si le parrain, de numéro $n$, est carré il attribue le numéro $2n$ à son filleul.
    \item Si le parrain est cube (de numéro $n$) il attribue le numéro $3n - 1$ à son filleul : un cube sait faire 2 opérations à la fois.
\end{itemize}

On peut remarquer que tous les entiers ne sont pas attribués : par exemple 7 ne peut pas être de la forme $2n$ ni de la forme $3n - 1$.

La question se pose de savoir si un entier est un nombre taupin possible : par exemple 1000 est obtenu par la suite
$1 \rightarrow 2 \rightarrow 5 \rightarrow 14 \rightarrow 28 \rightarrow 56 \rightarrow 167  \rightarrow 500  \rightarrow 1000$.


On peut remarquer que toutes les suites commencent par $1\rightarrow 2$.

%-------------------------------------------------------------------------------
%-------------------------------------------------------------------------------
\begin{Exercise}[title=Test de taupinalité]
Écrire une fonction \type{est\_taupin(n)} qui renvoie \type{true} ou \type{false} selon que $n$ est ou non un nombre taupin.
\end{Exercise}
%-------------------------------------------------------------------------------
\begin{Answer}
\begin{lstlisting}
let rec est_taupin n =
  if n <= 2
  then true
  else begin
    if n mod 2 = 0
    then begin
      if n mod 3 = 2 
      then est_taupin (n/2) || est_taupin ((n+1)/3)
      else est_taupin (n/2) end
    else begin
      if n mod 3 = 2
      then est_taupin ((n+1)/3)
      else false end end;;           
\end{lstlisting}
\newpage
\end{Answer}
%-------------------------------------------------------------------------------
%-------------------------------------------------------------------------------
\medskip
On aurait aimé pouvoir suivre la généalogie d'un nombre taupin, en calculant le nombre taupin du parrain. malheureusement il peut exister des nombres qui sont obtenus de plusieurs manières. Par exemple

$1 \rightarrow 2 \rightarrow 4 \rightarrow 8 \rightarrow 16 \rightarrow 32$ et
$1 \rightarrow 2 \rightarrow 4 \rightarrow 11 \rightarrow 32$.
%-------------------------------------------------------------------------------
%-------------------------------------------------------------------------------
\begin{Exercise}[title=Nombre de filières]
Écrire une fonction \type{filieres n} qui renvoie le nombre de manières (qui peut être 0) d'obtenir $k$ à partir de $2$ avec les transformations $k \mapsto 2k$ et $k \mapsto 3k-1$ pour tout $k$ compris entre 0 et $n$.
\end{Exercise}
%-------------------------------------------------------------------------------
\begin{Answer}
\begin{lstlisting}
let rec filieres n =
  if n <= 2
  then 1
  else begin
    if n mod 2 = 0
    then begin
      if n mod 3 = 2 
      then filieres (n/2) + filieres ((n+1)/3)
      else filieres (n/2) 
      end
    else begin
      if n mod 3 = 2
      then filieres ((n+1)/3)
      else 0
      end
    end;;           
\end{lstlisting}
\end{Answer}
%-------------------------------------------------------------------------------
%-------------------------------------------------------------------------------
\begin{Exercise}[title=3 filières]
Déterminer le premier nombre taupin qui peut peut l'être avec 3 chemins.
\end{Exercise}
%-------------------------------------------------------------------------------
\begin{Answer}
\begin{lstlisting}
let n = ref 1;;
while filieres !n < 4 do n := !n + 1 done;;
print_int !n;;
\end{lstlisting}

On trouve 20480
\end{Answer}
%-------------------------------------------------------------------------------
%-------------------------------------------------------------------------------
\begin{Exercise}[title=4 filières]
Déterminer le premier nombre taupin qui peut peut l'être avec 4 chemins.
\end{Exercise}
%-------------------------------------------------------------------------------
\begin{Answer}
Ici le calcul est long, 30 minutes. On peut accélérer en employant une file de priorité.
\begin{lstlisting}
let k_filieres k = 
  let rec aux nb n file =
    if nb = k
    then n
    else begin
      let n1 = next file in
      let new_f = add (add (remove file) (2*n1)) (3*n1 - 1) in
      if n1 = n
      then aux (nb + 1) n new_f
      else aux 1 n1 new_f end
  in aux 1 1 (add (createPQ()) 2);;
  \end{lstlisting}

On trouve $3\,988\,094\,144$ mais il faut 7 minutes !
\end{Answer}
%-------------------------------------------------------------------------------
%-------------------------------------------------------------------------------
%-------------------------------------------------------------------------------
%-------------------------------------------------------------------------------
\section{Propriétés des arbres fortement équilibrés}
%-------------------------------------------------------------------------------
%-------------------------------------------------------------------------------
Dans cette partie on suppose que \type{a} est un arbre fortement équilibré, c'est-à-dire qu'il vérifie la condition d'équilibre fort.
%-------------------------------------------------------------------------------
%-------------------------------------------------------------------------------
\begin{Exercise}{\it
\begin{enumerate}
\item Déterminer la taille des fils d'un arbre fortement équilibré $a$ de taille $n$.
\item Prouver qu'un arbre  fortement équilibré de taille $2^k-1$ est complet.
\item Prouver qu'un arbre  fortement équilibré de taille $n\ge 1$ avec $2^k \le n < 2^{k+1}$ est de hauteur $k$.
\item Prouver qu'un arbre  fortement équilibré est quasi-complet.
\item Montrer que la feuille la plus à gauche d'un arbre fortement équilibré est à la profondeur maximale.
\end{enumerate} }
\end{Exercise}
%-------------------------------------------------------------------------------
\begin{Answer} 
\begin{enumerate}
\item La taille des fils est $n_g$ et $n_d$ avec $n_d\le n_g \le n_d+1$ et $n_g+n_d+1=n$.

On a donc $2n_g\le n_g+n_d+1 \le n$ et $2n_g\ge n_g+n_d = n-1$.

Si $n$ est pair, $n=2p$, on a donc $n_g$ entier avec $p-\frac 12 \le n_g \le p$ puis $n_g =p$ puis $n_d=n-n-g-1=p-1$.

Si $n=2p+1$ est impair, alors $p \le n_g \le p+\frac 12$ donne $n_g=p$ et $n_d=p$.

On remarque qu'on a $n_g=\bigl\lfloor \frac n2\bigr\rfloor$.
\item On démontre le résultat par récurrence sur $k$.

Pour $k=1$ on a un arbre réduit à un seul nœud.

Si la propriété est vrai pour les arbres fortement équilibré de taille $2^{k-1}-1$ on conclut pour un arbre fortement équilibré de taille $2^k-1$ cas ses fils sont fortement équilibrés de taille $2^{k-1}-1$ donc complets.
\item On prouve le résultat par récurrence sur $n$.

Pour $n=1$, la hauteur est 0 et $2^0\le 1 < 2^1$.

Pour $n = 2$, la hauteur est 1 et $2^1\le 1 < 2^2$.

On suppose la propriété vraie pour tout arbre fortement équilibré de taille $m< n$.

On a $2^k \le n < 2^{k+1}$ donc $2^{k-1} \le n_g=\bigl\lfloor \frac n2\bigr\rfloor < 2^{k+1}$ donc le fils gauche est de hauteur $k-1$. Sauf dans le cas $n=2^k$, la taille du fils droit vérifie la même inégalité donc le fils droit est aussi de hauteur $k-1$. Ainsi l'arbre est de hauteur $k$ et ses deux fils sont de hauteur $k-1$.

Si $n=2^k$ son fils gauche est de hauteur $k-1$ et son fils droit est de hauteur $k-2$ donc l'arbre est de hauteur $k$.
\item Ici encore on procède par récurrence sur la taille.

On a vu ci-dessus que les deux fils d'un arbre fortement équilibré de hauteur $h$ sont de même hauteur $h-1$ sauf dans le cas d'une taille $2^h$. Dans ce dernier cas le fils droit est de taille $2^{h-1}-1$ donc est complet de hauteur $h-2$. Dans tous les cas les fils d'un arbre fortement équilibré de hauteur $h$ n'ont que des feuilles à la hauteur $h-1$ ou $h-2$ : l'arbre est quasi-complet.
\item Toujours une récurrence.
\end{enumerate}
\end{Answer}
%-------------------------------------------------------------------------------
%-------------------------------------------------------------------------------



