\documentclass[french,twoside]{book}
\usepackage[utf8]{inputenc}
\usepackage[T1]{fontenc}
\usepackage[french]{babel}
\usepackage{lmodern}
%-------------------------------------------------------------------------- 
% frenchb de babel change les points en tirets, retour à l'original :
\frenchbsetup{ItemLabeli=$\bullet$}
%--------------------------------------------------------------------------
% Utilitaires
%--------------------------------------------------------------------------
% Packages mathématiques
\usepackage{amsmath}
\usepackage{amsfonts}
\usepackage{amssymb}
\usepackage{amsthm}
\usepackage{bold-extra}
%--------------------------------------------------------------------------
%Graphismes
\usepackage{graphicx}%Pour l'inclusion d'images
\usepackage{tikz}%Pour les dessins divers (arbres, graphes...)
\usetikzlibrary{calc}
\usepackage{multicol}%pour du texte sur plusieurs colonnes
%--------------------------------------------------------------------------
%Pour mettre en page des exercices facilement 
\usepackage[lastexercise]{exercise}
\renewcommand{\ExerciseName}{Question}
\renewcommand{\ExerciseHeaderTitle}{\ ---\quad\textbf{\textsc{\ExerciseTitle}}}
\renewcommand{\ExerciseHeader}{\vspace{0cm}\noindent\textbf{\ExerciseName\ \ExerciseHeaderNB}\ \ExerciseHeaderDifficulty\ExerciseHeaderTitle\par}

\renewcommand{\AnswerHeader}{\vspace{2mm}\noindent\textbf{Solution\ ---\ }} 
\newcounter{exo}
\setcounter{exo}{0}
%--------------------------------------------------------------------------
\newif\ifcaml \camlfalse
\usepackage{listings}
% Configuration des listings
\lstset{escapechar = §,%
        frameround=tttt, frame = lines, linewidth=0.95\textwidth,%
        xleftmargin=0.025\textwidth,%
        breaklines=true,% 
        showstringspaces=false,%
       }
\lstset{
literate = {à}{{\`a}}1
           {À}{{\`A}}1
           {â}{{\^a}}1
           {ç}{{\c c}}1
           {Ç}{{\c C}}1
           {é}{{\'e}}1
           {É}{{\'E}}1
           {è}{{\`e}}1
           {È}{{\`E}}1
           {ê}{{\^e}}1
           {Ê}{{\^E}}1
           {ï}{{\"i}}1
           {î}{{\^i}}1
           {ô}{{\^o}}1
           {ù}{{\`u}}1
           {ÿ}{{\"y}}1
       }
\renewcommand{\lstlistingname}{Programme}
\lstset{keywordstyle=\bfseries,%
       commentstyle=\itshape,% 
       stringstyle=,% 
       basicstyle={\upshape\ttfamily}
      }
\ifcaml
  \lstset{tabsize = 2,%
          language=caml,%
         }
\else
  \lstset{tabsize = 4,%
          language=python,%
          morekeywords={as, assert, False, None, nonlocal, True, with, 
                        yield,float, int, abs, type}
          }
\fi
%--------------------------------------------------------------------------
\newtheoremstyle{definition}%
{}{}%
{\itshape}{}%
{\bfseries}{}%
{\newline}%
{\thmname{#1 : }\thmnumber{}\thmnote{#3}}
\theoremstyle{definition}
\newtheorem*{defin}{Définition}
%--------------------------------------------------------------------------
% Diverses fonctions
%--------------------------------------------------------------------------
\def\Vec#1{\overrightarrow{#1}}
\def\norme#1{\left\| #1 \right\|}
\def\C{{\Bbb C}}
\def\K{{\Bbb K}}
\def\M{{\cal M}}
\def\N{{\Bbb N}}
\def\Q{{\Bbb Q}}
\def\R{{\Bbb R}}
\def\Z{{\Bbb Z}}
\def\tr{{\rm tr}}
\def\trans#1{\,{\vphantom{#1}}^t\!#1\/}
\def\type#1{\text{\normalfont\ttfamily #1}}
\def\Type#1{{\normalfont\bfseries\ttfamily #1}}
\let\le=\leqslant 
% remplace le inférieur ou égal avec barre horizontale par une barre oblique
\let\ge=\geqslant 
% remplace le supérieur ou égal avec barre horizontale par une barre oblique
%--------------------------------------------------------------------------
% Présentation
%--------------------------------------------------------------------------
% Je n'aime pas les indentations de paragraphe :
\parindent=0pt
%--------------------------------------------------------------------------
%--------------------------------------------------------------------------
% Configuration de la page
%--------------------------------------------------------------------------
%--------------------------------------------------------------------------
%Taille des marges
\usepackage[a4paper,left=3cm,right=3cm,top=3cm,bottom=3cm]{geometry}
%--------------------------------------------------------------------------
% En-tête de chapitre (les différents poly)
\usepackage[explicit]{titlesec}
\titleformat{\chapter}[display]{\Large\bfseries}
            {\filright {\scshape\huge  D.S. \numero}}{0pt}
            {\titlerule \vspace{2ex}
             \filleft\sffamily\bfseries
               {\fontsize{40}{45}\usefont{T1}{qcs}{m}{sc}\selectfont #1}
            }
            [\vspace{2ex}\titlerule]

\titlespacing*{\chapter}{0pt}{0pt}{3cm}
%--------------------------------------------------------------------------
% Choix de la numérotation
\renewcommand{\thechapter}{\Roman{chapter}}
\renewcommand{\thesection}{\arabic{section}}
%--------------------------------------------------------------------------
%Configuration des en-têtes et bas-de-pages
\usepackage{fancyhdr}
% Le style pour les polys info sup
\pagestyle{fancy}
% Voici les commandes pour les intitulés leftmark, rightmark)
\renewcommand{\chaptermark}[1]{\markboth{\thechapter. {\sc #1}}{} }
\renewcommand{\sectionmark}[1]{\markright{\thechapter.\thesection-#1}{} }
% Hauts de page, O/E pour impair/pair (odd/even)
%                L/R pour gauche droite
\fancyhead[RO, LE]{DS \numero}% Pages impaires haut gauche : section
\fancyhead[LO,RE]{\classe}
% bas de page, mêmes positions
\fancyfoot[LO,RE]{}
\fancyfoot[LE,RO]{}
\fancyfoot[C]{\thepage}
\renewcommand{\headrulewidth}{0.2pt}% Ligne en haut
\renewcommand{\footrulewidth}{0pt}% pas en bas
%-------------------------------------------------------------------------- 
% Pour les premières pages, on peut imposer un style vide
% en écrivant *\thispagestyle{empty}* après le chapitre
%--------------------------------------------------------------------------
\newenvironment{abstract}{\begin{minipage}{0.2\linewidth} {\Large \bf Objectifs} \end{minipage}
\begin{minipage}{0.75\linewidth}\itshape} {\end{minipage}}
%--------------------------------------------------------------------------
% Personnalisations générales
\title{D.S. d'informatique}
\date{2019-2020}
\usepackage{enumitem}
\def\numero{03}
\def\classe{Option info MP1}

\camltrue
%-------------------------------------------------------------------------------
%-------------------------------------------------------------------------------
%-------------------------------------------------------------------------------
\begin{document}
%-------------------------------------------------------------------------------
%-------------------------------------------------------------------------------
%-------------------------------------------------------------------------------
\chapter{Couplages, mines 2012}
%-------------------------------------------------------------------------------
%-------------------------------------------------------------------------------
%-------------------------------------------------------------------------------
\section{Généralités}
%-------------------------------------------------------------------------------
%-------------------------------------------------------------------------------
%-------------------------------------------------------------------------------
\begin{figure}[h]
\begin{minipage}{.5\linewidth}
\centering
\begin{tikzpicture}[xscale=2,every node/.style={inner sep=0pt, minimum width = 16pt, circle, draw, fill=gray!30}]
\node (0a) at (0, 0) {$0_A$};
\node (1a) at (0, -1) {$1_A$};
\node (2a) at (0, -2) {$2_A$};
\node (3a) at (0, -3) {$3_A$};
\node (0b) at (1,  0) {$0_B$};
\node (1b) at (1, -1) {$1_B$};
\node (2b) at (1, -2) {$2_B$};
\node (3b) at (1, -3) {$3_B$}; 
\foreach \x/\y in {0/0, 0/1, 0/2, 1/3, 2/0, 2/1, 2/2, 2/3, 3/3} \draw (\x a) -- (\y b) ;
\end{tikzpicture}
\caption{Le graphe $G_0$.}
\end{minipage}
\begin{minipage}{.5\linewidth}
\centering
\begin{tikzpicture}[xscale=2,every node/.style={inner sep=0pt, minimum width = 16pt, circle, draw, fill=gray}]
\node (0a) at (0,0) {$0_A$};
\node (1a) at (0,-1) {$1_A$};
\node (2a) at (0,-2) {$2_A$};
\node (3a) at (0,-3) {$3_A$};
\node (0b) at (1,0) {$0_B$};
\node (1b) at (1,-1) {$1_B$};
\node (2b) at (1,-2) {$2_B$};
\node (3b) at (1,-3) {$3_B$}; 
\foreach \x/\y in {0/1, 0/2, 1/3, 2/0, 2/1, 2/2, 3/3} \draw (\x a) -- (\y b) ;
\foreach \x/\y in {0/0, 2/3} \draw[ultra thick, red] (\x a) -- (\y b) ;
\end{tikzpicture}
\caption{Le graphe $G_0$ et le couplage $C_0$.}
\end{minipage}
\end{figure}
%-------------------------------------------------------------------------------
%-------------------------------------------------------------------------------
\begin{Exercise}\it
Exhiber un couplage de cardinal 3 de $G_0$, puis indiquer s'il existe dans $G_0$ un couplage de cardinal 4. Justifier la réponse.
\end{Exercise}
%--------------------------------------------------------------------------
\begin{Answer}
$\bigl\{\{0_A,0_B\},\{1_A,3_B\},\{2_A,1_B\}\bigr\}$ est un couplage de cardinal 3 de $G_0$.

Il n'existe pas de couplage de cardinal 4 car les sommets $1_A$ et $3_A$ ne peuvent être couplés qu'avec $3_B$, ainsi un seul de ces deux sommet peut appartenir à couplage, tout couplage ne peut concerner que 3 sommets de $A$.
\end{Answer}
%-------------------------------------------------------------------------------
%-------------------------------------------------------------------------------
\begin{Exercise}\it
Écrire une fonction \type{verifie} telle que si \type{g} est une matrice codant le graphe $G$ et \type{c} un tableau codant le tableau $C$ alors \type{verifie g c} renvoie \type{true} si le tableau $C$ représente un couplage dans $G$ et \type{false} sinon.
Indiquer la complexité de la fonction \type{verifie}.
\end{Exercise}
%--------------------------------------------------------------------------
\begin{Answer}
Pour chaque nouveau couplage rencontré il faut vérifier si l'arête correspondante existe dans le graphe et si elle n'est pas incidente à une arête déjà rencontrée. Pour cela on utilise un tableau \type{vu} qui indique si un sommet de $B$ a déjà été rencontré.
\newpage
\begin{lstlisting}
let verifie g c =
  let n = Array.length c in
  let reponse = ref true in
  let vu = Array.make n false in
  for i = 0 to (n-1) do
    let j = c.(i) in
    if j <> -1 && not vu.(j)
    then begin
           reponse := !reponse && g.(i).(j);
           vu.(j) <- true end done;
  !reponse;;
\end{lstlisting}
La complexité est linéaire en $n$, le nombre de sommets.
\end{Answer}
%-------------------------------------------------------------------------------
%-------------------------------------------------------------------------------
\begin{Exercise}\it
Écrire une fonction \type{cardinal} telle que si \type{c} est un tableau codant un couplage alors \type{cardinal c} renvoie le cardinal de ce couplage.
Indiquer la complexité de la fonction \type{cardinal}.
\end{Exercise}
%--------------------------------------------------------------------------
\begin{Answer}
\begin{lstlisting}
let cardinal c = 
  let n = Array.length c in
  let reponse = ref 0 in
  for i = 0 to (n-1) do
    if c.(i) <> -1
    then reponse := !reponse +1 done;
  !reponse;;
\end{lstlisting}
La complexité est linéaire en $n$.
\end{Answer}
%-------------------------------------------------------------------------------
%-------------------------------------------------------------------------------
%-------------------------------------------------------------------------------
\section{Un couplage maximal}
%-------------------------------------------------------------------------------
%-------------------------------------------------------------------------------
%-------------------------------------------------------------------------------
\begin{Exercise}\it
Appliquer \type{algo\_approche} au graphe $G_0$.
\end{Exercise}
%--------------------------------------------------------------------------
\begin{Answer}
Les degrés dans $A$ sont $3,1,4,1$ ; les degrés dans $B$ sont $2,2,2,3$.
\begin{enumerate}
\item Les arêtes de somme minimum 4 sont $(1, 3)$ et $(3, 3)$. On choisit par exemple, $(1, 3)$.
\item Les degrés dans $A$ et $B$ sont alors $3,\_,3,0$ et $2,2,2,\_$.
\item Toutes les arêtes restantes ont pour somme 5. On choisit par exemple, $(0, 0)$.
\item Il ne reste que $(2 , 1)$ et $(2,2)$ de somme 3 chacune. On choisit par exemple, $(2,1)$.
\item Le graphe restant n'a plus d'arêtes.
\item On obtient ainsi $C=\{(1, 3), (0, 0), (2, 1)\}$ qui est bien maximal.
\end{enumerate}
\end{Answer}
%-------------------------------------------------------------------------------
\begin{figure}[h]
\centering
\begin{tikzpicture}[xscale=2,every node/.style={inner sep=0pt, minimum width = 16pt, circle, draw, fill=gray!30}]
\node (0a) at (0, 0) {$0_A$};
\node (1a) at (1, 0) {$1_A$};
\node (2a) at (2, 0) {$2_A$};
\node (3a) at (3, 0) {$3_A$};
\node (4a) at (4, 0) {$4_A$};
\node (5a) at (5, 0) {$5_A$};
\node (0b) at (0, 1) {$0_B$};
\node (1b) at (1, 1) {$1_B$};
\node (2b) at (2, 1) {$2_B$};
\node (3b) at (3, 1) {$3_B$}; 
\node (4b) at (4, 1) {$4_B$}; 
\node (5b) at (5, 1) {$5_B$}; 
\foreach \x/\y in {0/0, 0/1, 1/0, 1/1, 2/0, 2/1, 2/2, 3/2, 3/3, 4/3, 4/4, 4/5, 5/3, 5/4, 5/5} \draw (\x a) -- (\y b) ;
\end{tikzpicture}
\caption{Le graphe $G_1$.}
\end{figure}
%-------------------------------------------------------------------------------
%-------------------------------------------------------------------------------
\begin{Exercise}\it
Déterminer la première arête $a_1$ choisie par \type{algo\_approche} appliqué à $G_1$ ; 

tracer le graphe obtenu après suppression de $a_1$ et des arêtes incidentes à $a_1$. 

Montrer que le couplage obtenu par \type{algo\_approche} est de cardinal au plus $5$ 

et indiquer s'il est de cardinal maximal parmi les couplages de $G_1$.
\end{Exercise}
%--------------------------------------------------------------------------
\begin{Answer}
Dans $G_1$ toutes les arêtes ont une somme des degrés des extrémités égale à 4, 5 ou 6. Une seule a une somme égale à 4 : l'arête $a_1=\{3_A,2_B\}$. Une fois celle-ci éliminée ainsi que les arêtes incidentes il reste le graphe :
\begin{center}
\begin{tikzpicture}[xscale=2,every node/.style={inner sep=0pt, minimum width = 16pt, circle, draw}]
\node (0a) at (0, 0) {$0_A$};
\node (1a) at (1, 0) {$1_A$};
\node (2a) at (2, 0) {$2_A$};
\node (3a) at (3, 0) {$3_A$};
\node (4a) at (4, 0) {$4_A$};
\node (5a) at (5, 0) {$5_A$};
\node (0b) at (0, 1) {$0_B$};
\node (1b) at (1, 1) {$1_B$};
\node (2b) at (2, 1) {$2_B$};
\node (3b) at (3, 1) {$3_B$}; 
\node (4b) at (4, 1) {$4_B$}; 
\node (5b) at (5, 1) {$5_B$}; 
\foreach \x/\y in {0/0, 0/1, 1/0, 1/1, 2/0, 2/1, 4/3, 4/4, 4/5, 5/3, 5/4, 5/5} \draw (\x a) -- (\y b) ;
\end{tikzpicture}
\end{center}
Chacune de deux composantes connexes du graphe contient 2 couplages au maximum donc l'algorithme retournera un couplage de cardinal inférieur ou égal à 5. Or il existe un couplage de cardinal 6 : $\{\{0_A, 0_B\}, \{1_A, 1_B\}, \{2_A, 2_B\}, \{3_A, 3_B\}, \{4_A, 4_B\}, \{5_A, 5_B\}\}$ ; l'algorithme ne garantit donc pas d'obtenir un couplage de cardinal maximal.
\end{Answer}
%-------------------------------------------------------------------------------
%-------------------------------------------------------------------------------
\begin{Exercise}\it
Écrire une fonction \type{arete\_min} qui détermine une arête de $G$ dont la somme des degrés des extrémités soit minimum. Si le graphe possède au moins une arête, cette fonction modifie le tableau $a$ de deux entiers reçu en paramètre pour mettre dans les deux cases de $a$ les numéros des deux extrémités d'une arête qui atteint ce minimum ; dans ce cas la fonction renvoie la valeur \type{true} ; sinon elle renvoie la valeur \type{false}.
Indiquer la complexité de la fonction \type{arete\_min}.
\end{Exercise}
%--------------------------------------------------------------------------
\begin{Answer}
On commence par rédiger une fonction qui calcule les degrés des sommets de $A$ et des sommets de $B$ :
\begin{lstlisting}
let calcule_degre g =
  let n = Array.length g in
  let degA = Array.make n 0 in
  let degB = Array.make n 0 in
  for i = 0 to n-1 do
    for j = 0 to n-1 do
      if g.(i).(j) 
      then begin degA.(i) <- degA.(i) + 1; 
                 degB.(j) <- degB.(j) + 1 end done done;
  degA, degB;;
\end{lstlisting}
Cette première fonction est de complexité quadratique.

Il reste ensuite à partir à la recherche de l'arête minimale :
\begin{lstlisting}
let arete_min g a =
  let n = Array.length g in
  let degA, degB = calcule_degre g in
  let s = ref (2 * n + 1) in
  for i = 0 to n-1 do
    for j = 0 to n-1 do
      if g.(i).(j) && degA.(i) + degB.(j) < !s 
      then begin s := degA.(i) + degB.(j); 
                 a.(0) <- i; 
                 a.(1) <- j end
    done
  done;
  !s <= 2 * n;;
\end{lstlisting}
On notera que la somme des degrés des extrémités d'une arête ne peut excéder $2n$, ce qui explique la valeur initiale de la référence \type{s}.

Cette fonction est aussi de complexité quadratique.

\end{Answer}
%-------------------------------------------------------------------------------
\newpage
%-------------------------------------------------------------------------------
\begin{Exercise}\it
Écrire une fonction \type{supprimer} telle que si \type{g} est une matrice codant un graphe biparti équilibré $G$ et\type{a} un tableau de deux entiers codant une arête $a$ de $G$ alors \type{supprimer g a} modifie \type{g} pour que, après modifications, \type{g} code le graphe obtenu à partir de $G$ en supprimant $a$ ainsi que toutes les arêtes incidentes à $a$.
\end{Exercise}
%--------------------------------------------------------------------------
\begin{Answer}
\begin{lstlisting}
let supprimer g a =
  let n = Array.length g in
  let i = a.(0) and j = a.(1) in
  for k = 0 to n-1 do
    g.(i).(k) <- false; 
    g.(k).(j) <- false done;;
\end{lstlisting}
\end{Answer}
Cette fonction est de coût linéaire.

%-------------------------------------------------------------------------------
%-------------------------------------------------------------------------------
\begin{Exercise}\it
Écrire une fonction \type{algo\_approche} telle que si \type{g} est une matrice qui code un graphe biparti équilibré $G$, \type{algo\_approche} effectue \type{algo\_approche} à partir d'une copie de $G$ et renvoie un tableau codant le couplage obtenu.
Indiquer la complexité de la fonction \type{algo\_approche}.
\end{Exercise}
%--------------------------------------------------------------------------
\begin{Answer}
La fonction \type{Array.copy} réalise la copie d'un tableau, on peut d donc écrire
\begin{lstlisting}
let copy_matrix m = 
  let n = Array.length m in
  Array.init n (fun i -> Array.copy m.(i));;
\end{lstlisting}
On définit la fonction principale de l'algorithme \emph{algo\_approche} :
\begin{lstlisting}
let algo_approche g =
  let gg = copy_matrix g in
  let n = Array.length g in
  let c = Array.make n (-1) in
  let a = [| 0; 0 |] in
  while arete_min gg a do
    c.(a.(0)) <- a.(1);
    supprimer gg a done;
  c;;
\end{lstlisting}
Sachant que le coût de la fonction \type{arete\_min} est un ${\cal O}(n^2)$ et qu'un couplage maximal comporte au maximum $n$ couplages, le coût total de cet algorithme est en un ${\cal O}(n^3)$
\end{Answer}
%-------------------------------------------------------------------------------
\newpage
%-------------------------------------------------------------------------------
%-------------------------------------------------------------------------------
\section{Recherche exhaustive}
%-------------------------------------------------------------------------------
%-------------------------------------------------------------------------------
%-------------------------------------------------------------------------------
\begin{Exercise}\it
Écrire une fonction \type{une\_arete} telle que si \type{g} est une matrice codant un graphe $G$ et \type{a} un tableau de deux entiers alors \type{une\_arete g a} modifie le tableau \type{a} en y inscrivant les indice d'une arête dans le cas où $G$ possède au moins une arête.
\end{Exercise}
%--------------------------------------------------------------------------
\begin{Answer}
\begin{lstlisting}
let une_arete g a =
  let n = Array.length g in
  let trouve = ref false in
  let i = ref 0 in
  let j = ref 0 in
  while !i < 9 && not !trouve do
    if g.(!i).(!j)
    then begin a.(0) <- !i; 
               a.(1) <- !j; 
               trouve := true end
    else begin if j = 8 then (i := !i +1; j := 0)
                        else (j := !j + 1) end done;
  !trouve;;
\end{lstlisting}
\end{Answer}
%-------------------------------------------------------------------------------
%-------------------------------------------------------------------------------
\begin{Exercise}\it
Écrire une fonction récursive \type{meilleur\_couplage} telle que, si \type{g} est une matrice codant un graphe biparti équilibré $G$, \type{meilleur\_couplage g} renvoie un tableau codant un couplage de cardinal maximal dans $G$.
\end{Exercise}
%--------------------------------------------------------------------------
\begin{Answer}
Lorsque le graphe $G$ contient au moins une arête $a$, on réalise deux copies $G_1$ et $G_2$ de $G$. Dans la première on supprime l'arête $a$ et dans la seconde l'arête $a$ ainsi que toutes les arêtes incidentes.

Tout couplage $C_1$ de $G_1$ est un couplage de $G$ ne contenant pas l'arête $a$ ; tout couplage $C_2$ de $G_2$ à qui on ajoute $a$ est un couplage de $G$ contenant l'arête $a$. Ceci conduit à l'algorithme suivant :
\begin{lstlisting}
let rec meilleur_couplage g =
  let n = Array.length g in
  let a = [| 0; 0 |] in
  if une_arete g a 
  then begin
      let g1 = copy_matrix g in
      g1.(a.(0)).(a.(1)) <- false;
      let g2 = copy_matrix g in
      supprimer g2 a;
      let c1 = meilleur_couplage g1 in
      let c2 = meilleur_couplage g2 in
      c2.(a.(0)) <- a.(1);
      if cardinal c1 < cardinal c2 then c2 else c1
    end
  else Array.make n (-1);;
\end{lstlisting}

Les complexités temporelles et spatiales sont monstrueuses : de l'ordre de $n^2.2^{n^2}$.

\end{Answer}
%-------------------------------------------------------------------------------
%-------------------------------------------------------------------------------
\newpage
%-------------------------------------------------------------------------------
\section{L'algorithme hongrois}
%-------------------------------------------------------------------------------
%-------------------------------------------------------------------------------
%-------------------------------------------------------------------------------
\begin{figure}[h]
\centering
\begin{tikzpicture}[xscale=2,every node/.style={inner sep=0pt, minimum width = 16pt, circle, draw}]
\node (0a) at (0, 0) {$0_A$};
\node (1a) at (1, 0) {$1_A$};
\node (2a) at (2, 0) {$2_A$};
\node (3a) at (3, 0) {$3_A$};
\node (4a) at (4, 0) {$4_A$};
\node (5a) at (5, 0) {$5_A$};
\node (0b) at (0, 1) {$0_B$};
\node (1b) at (1, 1) {$1_B$};
\node (2b) at (2, 1) {$2_B$};
\node (3b) at (3, 1) {$3_B$}; 
\node (4b) at (4, 1) {$4_B$}; 
\node (5b) at (5, 1) {$5_B$}; 
\foreach \x/\y in {0/0, 0/1, 1/0, 1/1, 2/0, 2/1, 2/2, 3/2, 3/3, 4/3, 4/4, 4/5, 5/3, 5/4, 5/5} \draw (\x a) -- (\y b) ;
\foreach \x/\y in {0/0, 1/1, 3/2, 4/3, 5/5} \draw[very thick, red] (\x a) -- (\y b) ;
\end{tikzpicture}
\caption{Le graphe $G_1$ et le couplage $C_1$}
\end{figure}
%-------------------------------------------------------------------------------
%-------------------------------------------------------------------------------
\begin{Exercise}\it
Après avoir indiqué le seul sommet de $A$ qui puisse être l'origine d'une chaîne alternée augmentante relativement à $C_1$ et le seul sommet $B$ qui puisse être l'extrémité d'une chaîne alternée augmentante relativement à $C_1$, déterminer une chaîne alternée augmentante relativement à $C_1$.
\end{Exercise}
%--------------------------------------------------------------------------
\begin{Answer}
Le seul sommet non couplé de $A$ est le sommet $2_A$; il doit être au départ de toute chaîne alternée relativement à $C_1$.
Le seul sommet non couplé de $B$ est le sommet $4_B$ ; il doit être à l'arrivée de toute chaîne alternée augmentante relativement à $C_1$.

Une chaîne alternée augmentante peut être $2_A,2_B,3_A,3_B,4_A,4_B$.
\end{Answer}
%-------------------------------------------------------------------------------
%-------------------------------------------------------------------------------
\begin{Exercise}\it
On considère un graphe biparti équilibré $G$ et un couplage $C$ dans $G$. Montrer que s'il existe une chaîne alternée augmentante relativement à $C$ alors il existe dans $G$ un couplage dont le cardinal est égal au cardinal de $C$ augmenté de 1.
\end{Exercise}
%--------------------------------------------------------------------------
\begin{Answer}
Supposons que $(x_0 , x_1 , \ldots, x_{2p+1} )$ soit une chaîne alternée augmentante relativement à un couplage $C$.
Soit le couplage $C'$ obtenu à partir de $C$ en supprimant les $p$ arêtes $(x_{2k+1} , x_{2k+2} )$ pour $0 \le k < p$ et en ajoutant les $p+1$ arêtes ($x_{2k} , x_{2k+1} )$ pour $0 \le k \le p$. Les sommets $x_1, x_2, \ldots, x_{2p}$ ont perdu leur couplage en ôtant les arêtes de la forme $(x_{2k+1} , x_{2k+2} )$ donc on peut les coupler de nouveau en ajoutant aussi les sommets $x_0$ et $x_{2p+1}$.

Les arêtes ajoutées sont bien dans le graphe par définition d'une chaîne alternée donc on obtient bien  un couplage de graphe.
De plus son cardinal a été augmenté de 1.
\end{Answer}
%-------------------------------------------------------------------------------
\begin{figure}[h]
\centering
\begin{tikzpicture}[xscale=2,vrai/.style={inner sep=0pt, minimum width = 16pt, circle, draw}]
\node[vrai, fill=gray!30,label=below:0] (0a) at (0, 0) {$0_A$};
\node[vrai, fill=gray!30,label=below:-1] (1a) at (1, 0) {$1_A$} ;
\node[vrai, fill=gray!30,label=below:1] (2a) at (2, 0) {$2_A$};
\node[vrai, fill=gray!30,label=below:2] (3a) at (3, 0) {$3_A$};
\node[vrai,label=below:-1] (4a) at (4, 0) {$4_A$};
\node[vrai,label=below:-1] (5a) at (5, 0) {$5_A$};
\node[vrai, fill=gray!30,label=above:1] (0b) at (0, 1) {$0_B$};
\node[vrai, fill=gray!30,label=above:0] (1b) at (1, 1) {$1_B$};
\node[vrai, fill=gray!30,label=above:2] (2b) at (2, 1) {$2_B$};
\node[vrai, fill=gray!30,label=above:3] (3b) at (3, 1) {$3_B$}; 
\node[vrai,label=above:-1] (4b) at (4, 1) {$4_B$}; 
\node[vrai,label=above:-1] (5b) at (5, 1) {$5_B$}; 
\foreach \x/\y in {0/0, 0/1, 1/0, 1/1, 2/0, 2/1, 2/2, 3/2, 3/3, 4/3, 4/4, 4/5, 5/3, 5/4, 5/5} \draw (\x a) -- (\y b) ;
\foreach \x/\y in {0/0, 2/1, 3/2, 4/4, 5/5} \draw[very thick, red] (\x a) -- (\y b) ;
\end{tikzpicture}
\caption{Le graphe $G_1$ et le couplage $C'_1$, les sommets atteints sont grisés}
\end{figure}
%-------------------------------------------------------------------------------
%-------------------------------------------------------------------------------
\begin{Exercise}\it
Utiliser les marques pour reconstituer la chaîne alternée arrivant dans le sommet $3_B$ et correspondant aux marques.
Indiquer s'il s'agit d'une chaîne alternée augmentante relativement à $C'_1$.
\end{Exercise}
%--------------------------------------------------------------------------
\begin{Answer}
En partant de l'extrémité $3B$ , on remonte la chaîne grâce aux marques. On obtient ainsi la chaîne alternée $(1A , 0B , 0A , 1B , 2A , 2B , 3A , 3B )$. Comme $3B$ n'est pas couplé dans $C_1$ et appartient à $B$, il s'agit bien d'une chaîne alternée augmentante de $C_1$.

On augmente alors le couplage en
 $(0A , 1B ), (1A , 0B ), (2A , 2B ), (3A , 3B ), (4A , 4B ), (5A , 5B )$.
\end{Answer}
%-------------------------------------------------------------------------------
%-------------------------------------------------------------------------------
\begin{Exercise}\it
Écrire une fonction \type{actualiser} telle que si \type{c}, \type{r}, \type{mA}, \type{mB} sont des tableaux de $n$ entiers qui correspondent à un couplage, sa réciproque et les tableaux des marques d'une chaîne alternée augmentante et si \type{numero} est un entier donnant le numéro de $x_p$ alors \type{actualiser c r mA mB numero} modifie les tableaus \type{c} et \type{r} pour obtenir un couplage de cardinal égal à celui de $C$ augmenté de 1.
\end{Exercise}
%--------------------------------------------------------------------------
\begin{Answer}
On remonte la chaîne augmentante grâce aux marques en partant de l'extrémité $x_p$ sous la forme d'un indice $j$ tel que $x_p=j$.

\begin{itemize}
  \item On lit \texttt{i = mB[j]} ; on crée l'arête $(i,j)$ dans le couplage.
   \item Pour cela on place $i$ dans \texttt{R[j]} et $j$ dans \texttt{C[i]}
  \item On lit une nouvelle valeur de $j$ dans \texttt{mA[i]}
  \item On recommence si $j$ est différent de $-1$.
\end{itemize}


\begin{lstlisting}
let rec actualiser cpl inv mA mB numero =
  let i = mB.(numero) in 
  cpl.(i) <- numero;
  inv.(numero) <- i;
  let j = mA.(i) in
  if j <> -1
  then actualiser cpl inv mA mB j;;
\end{lstlisting}
\end{Answer}
%-------------------------------------------------------------------------------
%-------------------------------------------------------------------------------
\begin{figure}[ht]
\centering
\begin{tikzpicture}[xscale=2,every node/.style={inner sep=0pt, minimum width = 16pt, circle, draw}]
\node (0a) at (0, 0) {$0_A$};
\node (1a) at (1, 0) {$1_A$} ;
\node (2a) at (2, 0) {$2_A$};
\node (3a) at (3, 0) {$3_A$};
\node (4a) at (4, 0) {$4_A$};
\node (0b) at (0, 1) {$0_B$};
\node (1b) at (1, 1) {$1_B$};
\node (2b) at (2, 1) {$2_B$};
\node (3b) at (3, 1) {$3_B$}; 
\node (4b) at (4, 1) {$4_B$}; 
\foreach \x/\y in {0/0, 0/1, 0/2, 1/3, 1/4, 2/0, 2/1, 2/2, 2/3, 3/3, 3/4, 4/3, 4/4} \draw (\x a) -- (\y b);
\foreach \x/\y in {0/0, 2/2, 3/3, 4/4} \draw[very thick, red] (\x a) -- (\y b) ;
\end{tikzpicture}
\caption{Le graphe $G_2$ et le couplage $C_2$}
\end{figure}
%-------------------------------------------------------------------------------
%-------------------------------------------------------------------------------
\begin{Exercise}\it
Recopier la figure, encadrer tous les sommets qui peuvent être atteints et préciser à côté des sommets les marques obtenues. 

Indiquer s'il existe dans $G_2$ une chaîne alternée augmentante relativement à $C_2$.
\end{Exercise}
%--------------------------------------------------------------------------
\begin{Answer}
%-------------------------------------------------------------------------------
\begin{figure}[ht]
\centering
\begin{tikzpicture}[xscale=2,every node/.style={inner sep=0pt, minimum width = 16pt, circle, draw}]
\node[label=below:-1] (0a) at (0, 0) {$0_A$};
\node[fill=gray!30,label=below:-1] (1a) at (1, 0) {$1_A$} ;
\node[label=below:-1] (2a) at (2, 0) {$2_A$};
\node[fill=gray!30,label=below:3] (3a) at (3, 0) {$3_A$};
\node[fill=gray!30,label=below:4] (4a) at (4, 0) {$4_A$};
\node[label=above:-1] (0b) at (0, 1) {$0_B$};
\node[label=above:-1] (1b) at (1, 1) {$1_B$};
\node[label=above:-1] (2b) at (2, 1) {$2_B$};
\node[fill=gray!30,label=above:1] (3b) at (3, 1) {$3_B$}; 
\node[fill=gray!30,label=above:3] (4b) at (4, 1) {$4_B$}; 
\foreach \x/\y in {0/0, 0/1, 0/2, 1/3, 1/4, 2/0, 2/1, 2/2, 2/3, 3/3, 3/4, 4/3, 4/4} \draw (\x a) -- (\y b);
\foreach \x/\y in {0/0, 2/2, 3/3, 4/4} \draw[very thick, red] (\x a) -- (\y b) ;
\end{tikzpicture}
\caption{Le graphe $G_2$ et le premier marquage}
\end{figure}
%-------------------------------------------------------------------------------
%-------------------------------------------------------------------------------
\begin{figure}[ht]
\centering
\begin{tikzpicture}[xscale=2,every node/.style={inner sep=0pt, minimum width = 16pt, circle, draw}]
\node[label=below:-1] (0a) at (0, 0) {$0_A$};
\node[fill=gray!30,label=below:-1] (1a) at (1, 0) {$1_A$} ;
\node[label=below:-1] (2a) at (2, 0) {$2_A$};
\node[fill=gray!30,label=below:3] (3a) at (3, 0) {$3_A$};
\node[fill=gray!30,label=below:4] (4a) at (4, 0) {$4_A$};
\node[label=above:-1] (0b) at (0, 1) {$0_B$};
\node[label=above:-1] (1b) at (1, 1) {$1_B$};
\node[label=above:-1] (2b) at (2, 1) {$2_B$};
\node[fill=gray!30,label=above:4] (3b) at (3, 1) {$3_B$}; 
\node[fill=gray!30,label=above:1] (4b) at (4, 1) {$4_B$}; 
\foreach \x/\y in {0/0, 0/1, 0/2, 1/3, 1/4, 2/0, 2/1, 2/2, 2/3, 3/3, 3/4, 4/3, 4/4} \draw (\x a) -- (\y b);
\foreach \x/\y in {0/0, 2/2, 3/3, 4/4} \draw[very thick, red] (\x a) -- (\y b) ;
\end{tikzpicture}
\caption{Le graphe $G_2$ et le second marquage}
\end{figure}
%-------------------------------------------------------------------------------

Les deux seules chaînes alternées (maximales) sont $(1A , 3B , 3A , 4B , 4A )$ et 
$(1A , 4B , 4A , 3B , 3A )$ ; aucune n'est augmentante.
\end{Answer}
%-------------------------------------------------------------------------------
\newpage
%-------------------------------------------------------------------------------
\begin{Exercise}\it
Définir ce qu'on appelle récursivité croisée et indiquer comment elle peut être implémentée en \type{OCaml}, puis écrire les deux fonctions \type{chercherA} et \type{chercherB} ; chacune de ces deux fonctions reçoit en paramètres une matrice \type{g} codant le graphe $G$, les quatre tableaux \type{c}, \type{r}, \type{mA}, \type{mB} et un entier codant le numéro du sommet de départ de la recherche.

Ces deux fonctions modifient les tableaux \type{mA} et \type{mB} et renvoient le numéro d'un sommet non couplé de $B$ ou la valeur $-1$ selon le cas.
\end{Exercise}
%--------------------------------------------------------------------------
\begin{Answer}
La récursivité croisée consiste à déterminer en même temps deux fonctions dont chacune fait appel à l'autre. Elles sont définies par un même \type{let} et leurs définitions sont séparées par un \type{and}.

\medskip
On commence par chercher la liste des voisins d'un sommet de $A$:
\begin{lstlisting}
let voisinsB i g =
  let n = Array.length g in
  let l = ref [] in
  for j = 0 to (n-1) do
    if g.(i).(j) then l := j::(!l) done;
  !l;;
\end{lstlisting}

La recherche à partir de $B$ est simple : 

\begin{itemize}
  \item si le sommet est couplé, on suit le couplage
  \item sinon on a trouvé un point
\end{itemize}

Pour la recherche à partir de $A$ 

\begin{itemize}
  \item (2) : on définit une fonction auxiliaire qui parcourt une liste de sommets adjacents à $i$
  \item (3) : s'il n'y en a plus on renvoie -1
  \item (5) : si un sommet n'est pas couplé à $i$ et n'est pas marqué, on teste en (7)
  
        sinon on continue la recherche en (13)
  \item (7-8) : on le marque avec $i$ et on cherche (dans $B$) à partir de lui
  \item (9-10) : si cela n'aboutit pas on dé-marque le sommet et on continue la recherche
  \item (11) : si la recherche aboutit on a trouvé
\end{itemize}

\begin{lstlisting}[numbers=left]
let rec chercherA g c inv mA mB i =
  let rec aux liste =
    match liste with
    |[] -> -1
    |t::q -> if t <> c.(i) && mB.(t) = -1
             then begin mB.(t) <- i;
                  let rep = chercherB g c inv mA mB t in
                  if rep = -1
                  then (mB.(t) <- -1; aux q)
                  else rep end 
             else aux q in
  aux (voisinsB i g)
and chercherB g c inv mA mB j =
   match inv.(j) with
   |(-1) -> j
   |i -> mA.(i) <- j;
         chercherA g c inv mA mB i;;
\end{lstlisting}
\end{Answer}
%-------------------------------------------------------------------------------
\newpage
%-------------------------------------------------------------------------------
\begin{Exercise}\it
Écrire une fonction \type{chaine\_alternee} telle que si \type{g} est une matrice codant le graphe $G$, \type{c}, \type{r}, \type{mA} et \type{mB} des tableaux, toutes les cases de \type{mA} et \type{mB} étant initialisées à $-1$, alors \type{chaine\_alternee g c r mA mB} renvoie :
\begin{itemize}
  \item $-1$ s'il n'existe pas de chaîne alternée augmentante ;
  \item le numéro de l'extrémité d'une chaîne alternée augmentante dans le cas contraire.
\end{itemize}
De plus, la fonction modifie les tableaux \type{mA} et \type{mB} pour qu'ils contiennent les marques des sommets à la fin de l'exécution de la fonction.
\end{Exercise}
%--------------------------------------------------------------------------
\begin{Answer}
On parcourt dans l'ordre les sommets de $A$ jusqu'à trouver un éventuel sommet non couplé origine d'une chaîne
alternée augmentante. Ne pas oublier de ré-initialiser les tableaux de marquage à chaque étape.

\begin{lstlisting}
let chaine_alternee g c inv mA mB =
  let n = Array.length c in
  let rec aux i =
    if i = n
    then -1
    else begin for k = 0 to (n-1) do mA.(k) <- -1 done;
        for k = 0 to (n-1) do mB.(k) <- -1 done;
        let j = chercherA g c inv mA mB i in
        if j = -1 then aux (i+1) else j end in
  aux 0;;
\end{lstlisting}
\end{Answer}
%-------------------------------------------------------------------------------
%-------------------------------------------------------------------------------
\begin{Exercise}\it
Écrire  la fonction \type{algorithme\_hongrois} telle que si \type{g} est une matrice codant un graphe biparti équilibré $G$, alors \type{algorithme\_hongrois g} renvoie un tableau codant le couplage obtenu par l'algorithme hongrois.
\end{Exercise}
%--------------------------------------------------------------------------
\begin{Answer}
Le programme consiste  
\begin{itemize}
  \item à partir d'un couplage (le premier étant le couplage vide)
  \item on marque les sommets (\type{mA} et \type{mB}) avec -1
  \item on cherche une chaîne alternée augmentante avec \type{chaine\_alternee}, cela modifie \type{mA} et \type{mB}
  \item s'il en existe une on augmente de couplage avec \type{actualiser}, cela modifie \type{c} et \type{inv} et on recommence
\end{itemize}
L'algorithme termine car a chaque tour de boucle le cardinal du couplage augmente de une unité : la boucle effectue au plus $n$ passages.
\begin{lstlisting}
let algorithme_hongrois g = 
  let n = Array.length g in
  let c = Array.make n (-1) in
  let inv = Array.make n (-1) in
  let encore = ref true in
  while !encore do
    let mA = Array.make n (-1) 
    and mB = Array.make n (-1) in
    let num = chaine_alternee g c inv mA mB in
    if (num <> -1) 
    then actualiser c inv mA mB num
    else encore := false done;
  c;;
\end{lstlisting}
\end{Answer}
%-------------------------------------------------------------------------------
%-------------------------------------------------------------------------------


%-------------------------------------------------------------------------------
%-------------------------------------------------------------------------------
%-------------------------------------------------------------------------------
\end{document}





\question
On suppose donnés un graphe biparti équilibré $G$ d'ordre $2n$ et un couplage $C$ dans $G$. Tous les sommets de $G$ possèdent une marque égale à $-1$.

La fonction \type{chaîne\_alternee} cherche s'il existe une chaîne alternée augmentante en appliquant la fonction \type{chercherA} successivement à partir des sommets non couplés de $A$.


\question
Dans cette question on programme l'algorithme hongrois.



\end{document}
