\documentclass[french,twoside]{book}
\usepackage[utf8]{inputenc}
\usepackage[T1]{fontenc}
\usepackage[french]{babel}
\usepackage{lmodern}
%-------------------------------------------------------------------------- 
% frenchb de babel change les points en tirets, retour à l'original :
\frenchbsetup{ItemLabeli=$\bullet$}
%--------------------------------------------------------------------------
% Utilitaires
%--------------------------------------------------------------------------
% Packages mathématiques
\usepackage{amsmath}
\usepackage{amsfonts}
\usepackage{amssymb}
\usepackage{amsthm}
\usepackage{bold-extra}
%--------------------------------------------------------------------------
%Graphismes
\usepackage{graphicx}%Pour l'inclusion d'images
\usepackage{tikz}%Pour les dessins divers (arbres, graphes...)
\usetikzlibrary{calc}
\usepackage{multicol}%pour du texte sur plusieurs colonnes
%--------------------------------------------------------------------------
%Pour mettre en page des exercices facilement 
\usepackage[lastexercise]{exercise}
\renewcommand{\ExerciseName}{Question}
\renewcommand{\ExerciseHeaderTitle}{\ ---\quad\textbf{\textsc{\ExerciseTitle}}}
\renewcommand{\ExerciseHeader}{\vspace{0cm}\noindent\textbf{\ExerciseName\ \ExerciseHeaderNB}\ \ExerciseHeaderDifficulty\ExerciseHeaderTitle\par}

\renewcommand{\AnswerHeader}{\noindent\textbf{Solution de la question \ExerciseHeaderNB{} - }} 
\newcounter{exo}
\setcounter{exo}{0}
%--------------------------------------------------------------------------
\newif\ifcaml \camlfalse
\usepackage{listings}
% Configuration des listings
\lstset{escapechar = §,%
        frameround=tttt, frame = lines, linewidth=0.95\textwidth,%
        xleftmargin=0.025\textwidth,%
        breaklines=true,% 
        showstringspaces=false,%
       }
\lstset{
literate = {à}{{\`a}}1
           {À}{{\`A}}1
           {â}{{\^a}}1
           {ç}{{\c c}}1
           {Ç}{{\c C}}1
           {é}{{\'e}}1
           {É}{{\'E}}1
           {è}{{\`e}}1
           {È}{{\`E}}1
           {ê}{{\^e}}1
           {Ê}{{\^E}}1
           {ï}{{\"i}}1
           {î}{{\^i}}1
           {ô}{{\^o}}1
           {œ}{{\oe}}1
           {ù}{{\`u}}1 
           {ÿ}{{\"y}}1
       }
\renewcommand{\lstlistingname}{Programme}
\lstset{keywordstyle=\bfseries,%
       commentstyle=\itshape,% 
       stringstyle=,% 
       basicstyle=\ttfamily
      }
\ifcaml
  \lstset{tabsize = 2,%
          language=caml,%
         }
\else
  \lstset{tabsize = 4,%
          language=python,%
          morekeywords={as, assert, False, None, nonlocal, True, with, 
                        yield,float, int, abs, type}
          }
\fi
%--------------------------------------------------------------------------
\newtheoremstyle{definition}%
{}{}%
{\itshape}{}%
{\bfseries}{}%
{\newline}%
{\thmname{#1 : }\thmnumber{}\thmnote{#3}}
\theoremstyle{definition}
\newtheorem*{defin}{Définition}
\newtheorem*{thm}{Théorème}
%--------------------------------------------------------------------------
% Diverses fonctions
%--------------------------------------------------------------------------
\def\Vec#1{\overrightarrow{#1}}
\def\norme#1{\left\| #1 \right\|}
\def\C{{\mathbb C}}
\def\K{{\mathbb K}}
\def\M{{\cal M}}
\def\N{{\mathbb N}}
\def\Q{{\mathbb Q}}
\def\R{{\mathbb R}}
\def\Z{{\mathbb Z}}
\def\tr{{\rm tr}}
\def\trans#1{\,{\vphantom{#1}}^t\!#1\/}
\def\type#1{\text{\normalfont\ttfamily #1}}
\def\Type#1{{\normalfont\bfseries\ttfamily #1}}
\let\le=\leqslant 
% remplace le inférieur ou égal avec barre horizontale par une barre oblique
\let\ge=\geqslant 
% remplace le supérieur ou égal avec barre horizontale par une barre oblique
%--------------------------------------------------------------------------
% Présentation
%--------------------------------------------------------------------------
% Je n'aime pas les indentations de paragraphe :
\parindent=0pt
%--------------------------------------------------------------------------
%--------------------------------------------------------------------------
% Configuration de la page
%--------------------------------------------------------------------------
%--------------------------------------------------------------------------
%Taille des marges
\usepackage[a4paper,left=3cm,right=3cm,top=3cm,bottom=3cm]{geometry}
%--------------------------------------------------------------------------
% En-tête de chapitre (les différents poly)
\usepackage[explicit]{titlesec}
\titleformat{\chapter}[display]{\Large\bfseries}
            {\filright {\scshape\huge  D.S. \numero}}{0pt}
            {\titlerule \vspace{2ex}
             \filleft\sffamily\bfseries
               {\fontsize{50}{55}\usefont{T1}{qcs}{m}{sc}\selectfont #1}
            }
            [\vspace{2ex}\titlerule]

\titlespacing*{\chapter}{0pt}{0pt}{3cm}
%--------------------------------------------------------------------------
% Choix de la numérotation
\renewcommand{\thechapter}{\Roman{chapter}}
\renewcommand{\thesection}{\arabic{section}}
%--------------------------------------------------------------------------
%Configuration des en-têtes et bas-de-pages
\usepackage{fancyhdr}
% Le style pour les polys info sup
\pagestyle{fancy}
% Voici les commandes pour les intitulés leftmark, rightmark)
\renewcommand{\chaptermark}[1]{\markboth{\thechapter. {\sc #1}}{} }
\renewcommand{\sectionmark}[1]{\markright{\thechapter.\thesection-#1}{} }
% Hauts de page, O/E pour impair/pair (odd/even)
%                L/R pour gauche droite
\fancyhead[RO, LE]{DS \numero}% Pages impaires haut gauche : section
\fancyhead[LO,RE]{\classe}
% bas de page, mêmes positions
\fancyfoot[LO,RE]{}
\fancyfoot[LE,RO]{}
\fancyfoot[C]{\thepage}
\renewcommand{\headrulewidth}{0.2pt}% Ligne en haut
\renewcommand{\footrulewidth}{0pt}% pas en bas
%-------------------------------------------------------------------------- 
\renewcommand{\thefigure}{\arabic{figure}}
%-------------------------------------------------------------------------- 
% Pour les premières pages, on peut imposer un style vide
% en écrivant *\thispagestyle{empty}* après le chapitre
%--------------------------------------------------------------------------
\newenvironment{abstract}{\begin{minipage}{0.2\linewidth} {\Large \bf Objectifs} \end{minipage}
\begin{minipage}{0.75\linewidth}\itshape} {\end{minipage}}
%--------------------------------------------------------------------------
% Personnalisations générales
\title{D.S. d'informatique}
\date{2018-2019}
