\input{styleDS}
\usepackage{enumitem}
\def\numero{02+}
\def\classe{Option info MP1}

\camltrue
%-------------------------------------------------------------------------------
%-------------------------------------------------------------------------------
%-------------------------------------------------------------------------------
\begin{document}
%-------------------------------------------------------------------------------
%-------------------------------------------------------------------------------
%-------------------------------------------------------------------------------
\chapter{Détection de carrés dans les mots,\\ ENS info-math 2017}
%-------------------------------------------------------------------------------
%-------------------------------------------------------------------------------
%-------------------------------------------------------------------------------
\section{Existence d'un carré}
%-------------------------------------------------------------------------------
%-------------------------------------------------------------------------------
%---01--------------------------------------------------------------------------
\begin{Exercise}
\begin{lstlisting}
let carre_pe_pos m p i =
  let n = Array.length m in
  let j = ref 0 in
  while !j < p && i + !j + p  < n 
               && m.(i + !j + p) = m.(i + !j)
    do j := !j + 1 done;
  !j;;
  
carre_pe_pos [|'c'; 'o'; 'u'; 'c'; 'o'; 'u' |] 3 0;;
3
carre_pe_pos [|'c'; 'o'; 'u'; 'c'; 'o'; 'o' |] 3 0;;
2
carre_pe_pos [|'a'; 'b'; 'a'; 'b'; 'a'; 'b'; 'a'; 'b'|] 2 4;;
2
carre_pe_pos [|'a'; 'b'; 'a'; 'b'; 'a'; 'b'; 'a'; 'b'|] 2 5;;
1
\end{lstlisting}
\end{Exercise}
%-------------------------------------------------------------------------------
%---02--------------------------------------------------------------------------
\begin{Exercise}
La fonction renvoie le nombre maximal, majoré par $p$, de lettres successives de $m$ commençant à la position $i$ et égales à la lettre décalée à droite de $p$ positions.

Si la valeur renvoyée est $p$ c'est que $m$ contient un carré de période $p$ à la position $i$.
\end{Exercise}
%-------------------------------------------------------------------------------
\newpage
%---03--------------------------------------------------------------------------
\begin{Exercise}
On teste toutes les positions pour trouver une réponse $p$ à  \type{carre\_pe\_pos m p i}. Les positions possibles pour trouver un carré de période $p$ vont de 0 à $|m| - 2 p$
\begin{lstlisting}
let carre_pe m p =
  let i_max = Array.length m - 2*p in
  let rec aux i =
    if i > i_max 
    then false
    else if carre_pe_pos m p i = p
         then true
         else aux (i+1)
  in aux 0;;
\end{lstlisting}
\end{Exercise}
%-------------------------------------------------------------------------------
%---04--------------------------------------------------------------------------
\begin{Exercise} Pour chercher un carré, on teste toutes les périodes possibles. On doit commencer par la période 1 et non pas 0.
\begin{lstlisting}
let carre_naif m =
  let p_max = Array.length m/2 in
  let rec aux p =
    if p > p_max
    then false
    else if carre_pe r p
         then true
         else aux (p+1)
  in aux 1;;
\end{lstlisting}
\end{Exercise}
%-------------------------------------------------------------------------------
%---05--------------------------------------------------------------------------
\begin{Exercise} Si un mot ne contient pas de carré, on fait tous les tests. On compte le nombre de tests (les comparaisons de lettres).

\begin{itemize}
  \item La fonction \type{carre\_pe\_pos} effectue au plus $p$ comparaisons.
  \item La fonction \type{carre\_pe} appelle \type{carre\_pe\_pos} pour $p$ variant de 0 à $n-2p$ ; le nombre de comparaisons est majoré par $n^2$.
  \item La fonction \type{carre\_naif} appelle \type{carre\_pe} $n/2$ fois ; le nombre de comparaisons est majoré par $n^3$.
\end{itemize}
La complexité est cubique.
\end{Exercise}
%-------------------------------------------------------------------------------
%---06--------------------------------------------------------------------------
\begin{Exercise}
\end{Exercise}
%-------------------------------------------------------------------------------
%---07--------------------------------------------------------------------------
\begin{Exercise}
\end{Exercise}
%-------------------------------------------------------------------------------
%---08--------------------------------------------------------------------------
\begin{Exercise}
\end{Exercise}
%-------------------------------------------------------------------------------
%---09--------------------------------------------------------------------------
\begin{Exercise}
\end{Exercise}
%-------------------------------------------------------------------------------
%---10--------------------------------------------------------------------------
\begin{Exercise}
\end{Exercise}
%-------------------------------------------------------------------------------
%-------------------------------------------------------------------------------
%-------------------------------------------------------------------------------
\section{Introduction}
%-------------------------------------------------------------------------------
%-------------------------------------------------------------------------------
%---11--------------------------------------------------------------------------
\begin{Exercise}
\end{Exercise}
%-------------------------------------------------------------------------------
%---12--------------------------------------------------------------------------
\begin{Exercise}
\end{Exercise}
%-------------------------------------------------------------------------------
%---13--------------------------------------------------------------------------
\begin{Exercise}
\end{Exercise}
%-------------------------------------------------------------------------------
%---14--------------------------------------------------------------------------
\begin{Exercise}
\end{Exercise}
%-------------------------------------------------------------------------------
%-------------------------------------------------------------------------------
%-------------------------------------------------------------------------------
\section{Introduction}
%-------------------------------------------------------------------------------
%-------------------------------------------------------------------------------
%---15--------------------------------------------------------------------------
\begin{Exercise}
\end{Exercise}
%-------------------------------------------------------------------------------
%---16--------------------------------------------------------------------------
\begin{Exercise}
\end{Exercise}
%-------------------------------------------------------------------------------
%---17--------------------------------------------------------------------------
\begin{Exercise}
\end{Exercise}
%-------------------------------------------------------------------------------
%---18--------------------------------------------------------------------------
\begin{Exercise}
\end{Exercise}
%-------------------------------------------------------------------------------
%---19--------------------------------------------------------------------------
\begin{Exercise}
\end{Exercise}
%-------------------------------------------------------------------------------
%---20--------------------------------------------------------------------------
\begin{Exercise}
\end{Exercise}
%-------------------------------------------------------------------------------
%---21--------------------------------------------------------------------------
\begin{Exercise}
\end{Exercise}
%-------------------------------------------------------------------------------
%---22--------------------------------------------------------------------------
\begin{Exercise}
\end{Exercise}
%-------------------------------------------------------------------------------
%-------------------------------------------------------------------------------
\end{document}
%--------------------------------------------------------------------------
\begin{Answer}
\begin{lstlisting}
\end{lstlisting}
\end{Answer}
%-------------------------------------------------------------------------------
%-------------------------------------------------------------------------------
\medskip
