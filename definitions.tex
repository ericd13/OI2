%-------------------------------------------------------------------------- 
%-------------------------------------------------------------------------- 
%-------------------------------------------------------------------------- 
% On utilise la classe book, pour avoir les chapitres
\documentclass[french]{book}
%-------------------------------------------------------------------------- 
%-------------------------------------------------------------------------- 
% Francisation des règles typographiques
\usepackage[french]{babel}
% frenchb change les points en tirets, retour à l'original :
\frenchbsetup{ItemLabeli=$\bullet$}
%-------------------------------------------------------------------------- 
%-------------------------------------------------------------------------- 
% Fontes
\usepackage[utf8]{inputenc}
\usepackage[T1]{fontenc}
\usepackage{lmodern}
%--------------------------------------------------------------------------
%--------------------------------------------------------------------------
% Utilitaires
%-------------------------------------------------------------------------- 
% Packages mathématiques
\usepackage{amsmath}
\usepackage{amsfonts}
\usepackage{amssymb}
\usepackage{amsthm}
%--------------------------------------------------------------------------
%Graphismes
\usepackage{graphicx}%Pour l'inclusion d'images
\graphicspath{{./Images/}}
\usepackage{wrapfig}%Figures sur les côtés
%--------------------------------------------------------------------------
\usepackage{tikz}%Pour les dessins divers (arbres, graphes...)
\usetikzlibrary{calc, positioning, patterns ,babel, shapes, automata}
\usepackage[european, cuteinductors, EFvoltages]{circuitikz}
%-------------------------------------------------------------------------- 
%--------------------------------------------------------------------------
% Présentations des pages 
%--------------------------------------------------------------------------
%Taille des marges
\usepackage[a4paper,left=3cm,right=3cm,top=3cm,bottom=3cm]{geometry}
%--------------------------------------------------------------------------
% En-tête de chapitre
\usepackage[explicit]{titlesec}
\titleformat{\chapter}[display]{\Large\bfseries}
            {\filright {\chaptername}\ \huge\thechapter }{0pt}
            {\titlerule \vspace{2ex}
             \filleft\sffamily\bfseries
               {\fontsize{40}{45}\usefont{T1}{qcs}{m}{sc}\selectfont #1}
            }
            [\vspace{2ex}\titlerule]
\titlespacing*{\chapter}{0pt}{0pt}{3cm}
%--------------------------------------------------------------------------
%Liens
\usepackage[colorlinks=true,urlcolor=blue,linkcolor=blue]{hyperref}
%--------------------------------------------------------------------------
% Choix de la numérotation
\renewcommand{\thechapter}{\Roman{chapter}}
\renewcommand{\thesection}{\arabic{section}}
%--------------------------------------------------------------------------
% Une table des matières mieux présentée
\usepackage{tocloft}
\renewcommand\cftchapnumwidth{1cm}
%--------------------------------------------------------------------------
% Je n'aime pas les indentations de paragraphe :
\parindent=0pt
%--------------------------------------------------------------------------
%Configuration des en-têtes et bas-de-pages
\usepackage{fancyhdr}
\pagestyle{fancy}
% Voici les commandes pour les intitulés leftmark, rightmark)
\renewcommand{\chaptermark}[1]{\markboth{ {\sc #1}}{} }
\renewcommand{\sectionmark}[1]{\markright{\thesection-#1}{} }
% Hauts de page, O/E pour impair/pair (odd/even)
%                L/R pour gauche droite
\fancyhead[RO]{\rightmark}% Pages impaires haut gauche : section
\fancyhead[LE]{\leftmark}% Pages paires haut droit : chapitre
\fancyhead[LO,RE]{}
% bas de page, mêmes positions
\fancyfoot[LO,RE]{\annee}
\fancyfoot[LE,RO]{\classe}
\fancyfoot[C]{\thepage}
\renewcommand{\headrulewidth}{0.2pt}% Ligne en haut
\renewcommand{\footrulewidth}{0pt}% pas en bas
%-------------------------------------------------------------------------- 
% Pour les premières pages, on peut imposer un style vide
% en écrivant *\thispagestyle{empty}* après le chapitre
%-------------------------------------------------------------------------- 
% Pour du texte sur plusieurs colonnes
\usepackage{multicol}%
%-------------------------------------------------------------------------- 
% Pour diminuer la séparation entre items des listes
\usepackage{enumitem}
\setlist{nosep} % or \setlist{noitemsep} 
%--------------------------------------------------------------------------
% Profondeur de la table des matières : 1 = section
\setcounter{tocdepth}{1}   
%--------------------------------------------------------------------------
% Définition de la page de titre
\def\premierePage{\begin{titlepage}
              {\scshape\LARGE Lycée Faidherbe, \annee \par}
              \vspace{3cm}
              \begin{center}
                  \rule{\linewidth}{0.5mm}
                  \vskip -2cm
                  \rule{\linewidth}{1.5mm}
                  {\fontsize{60}{65}\usefont{T1}{qcs}{m}{sc}\selectfont \titre\par}
                  \rule{\linewidth}{1.5mm}
                  \vskip -2mm
                  \rule{\linewidth}{0.5mm}
                  
                  \vspace{3cm}
                  {\Large\bf \classe}
              \end{center}
              \vfill
              {\large Version du \today\par}
           \end{titlepage}
          }
%-------------------------------------------------------------------------- 
%--------------------------------------------------------------------------
% Composants
%-------------------------------------------------------------------------- 
%--------------------------------------------------------------------------
%Pour mettre en page des exercices facilement 
\usepackage[answerdelayed,lastexercise]{exercise}
\renewcommand{\ExerciseName}{Exercice}
\renewcommand{\ExerciseHeaderTitle}{\ ---\quad\textbf{\textsc{\ExerciseTitle}}}
\renewcommand{\ExerciseHeader}{\vspace{0cm}\noindent
                               \textbf{\ExerciseName\ \ExerciseHeaderNB}\ 
                               \ExerciseHeaderDifficulty
                               \ExerciseHeaderTitle\par
                              }
\renewcommand{\AnswerHeader}{\noindent\textbf{Solution de l'exercice \ExerciseHeaderNB{} - }} 
% \numberwithin{Answer}{chapter}
% \numberwithin{Exercise}{chapter}
\renewcounter{Exercise}[chapter]
\setlength{\ExerciseSkipBefore}{0.6\baselineskip}
\setlength{\ExerciseSkipAfter}{0.6\baselineskip}
%--------------------------------------------------------------------------
% Configuration des listings
\definecolor{mon_orange}{RGB}{165,75,0}
\definecolor{mon_bleu}{RGB}{16,48,255}
\definecolor{mon_vert}{RGB}{16,165,0}
\usepackage{listings}
\lstset{escapechar = §,%
%        frameround=tttt, frame = lines, linewidth=0.95\textwidth,%
        xleftmargin=0.025\textwidth,%
        breaklines=true,% 
        showstringspaces=false,%
       }
\lstset{
literate = {à}{{\`a}}1
           {À}{{\`A}}1
           {â}{{\^a}}1
           {ç}{{\c c}}1
           {Ç}{{\c C}}1
           {é}{{\'e}}1
           {É}{{\'E}}1
           {è}{{\`e}}1
           {È}{{\`E}}1
           {ê}{{\^e}}1
           {Ê}{{\^E}}1
           {ï}{{\"i}}1
           {î}{{\^i}}1
           {ô}{{\^o}}1
           {ù}{{\`u}}1
           {ÿ}{{\"y}}1
       }
\renewcommand{\lstlistingname}{Programme}
\lstset{tabsize = 4,%
        language=caml,%
       }
\lstset{emph={List, Array, String},
        morekeywords={when, failwith},
        keywordstyle=\color{mon_orange},%
        emphstyle=\bfseries\color{mon_bleu},%
        commentstyle=\slshape,% 
        stringstyle=\color{mon_vert},% 
        basicstyle=\ttfamily\upshape
       }
\lstMakeShortInline@
\lstdefinestyle{ocaml}{language = caml,
                       tabsize=3,
                       morekeywords = {failwith, false, not, ref, true, when},
                       morekeywords = [2]{print_int, print_string, print_float, print_newline},
                       emph={List, Array, String},
                       basicstyle=\ttfamily\upshape,
                       emphstyle=\bfseries\color{mon_bleu},
                       commentstyle=\slshape,
                       stringstyle=\color{mon_vert},
                       keywordstyle=\color{mon_orange},,
                       keywordstyle=[2]\color{mon_orange!50}}
%--------------------------------------------------------------------------
% Environnements 
\usepackage[skins, listings, theorems, xparse]{tcolorbox}
%---------------Définitions------------------------------------------------
\newtcbtheorem[no counter]
              {defin}% Nom de l'environnement
              {} % pas de nom
              {enhanced,
               before skip=6mm,
               colback=green!5,
               colbacktitle=green!35!black,
               colframe=green!35!black,
               separator sign = {},
               attach boxed title to top left = {},
               fonttitle=\bfseries}
              {def}
%---------------Définitions de notation---------------------------------
\def\poser#1#2{
\begin{tcolorbox}[colback=blue!5,
                  colframe=blue!35!black, 
                  sidebyside, 
                  lefthand width = 15mm,
                  lower separated=false,
                  sidebyside gap=0mm]
#1
\tcblower
#2
\end{tcolorbox}}

%---------------Code en passant-----------------------------------------
\DeclareTCBListing[no counter]
                  {ocaml}
                  { o }
                  {listing only,
                   before skip=2mm,
                   colback=yellow!5,
                   colframe=yellow!40!black,
                   top = -2mm,
                   left=-2mm,
                   right=-3mm,
                   bottom=-2mm,
                   listing options = {style=ocaml},
                   IfValueTF={#1}{title={#1}}{}
                  }

%---------------Code titré numéroté-----------------------------------------
\DeclareTCBListing[auto counter,
                   number within=chapter]
                  {code}
                  { m }
                  {listing only,
                   colback=yellow!5,
                   colframe=yellow!40!black,
                   before skip=4mm,
                   halign title=center,
                   top = -1mm,
                   right= -3mm,
                   left=-1mm,
                   bottom=-1mm,
                   listing options = {style=ocaml},
                   fonttitle=\bfseries,
                   title={Code \thetcbcounter : #1}
                  }

%--------------------------------------------------------------------------
% Diverses fonctions
%--------------------------------------------------------------------------
\def\C{{\mathbb C}}
\def\K{{\mathb K}}
\def\M{{\cal M}}
\def\N{{\mathbb N}}
\def\Q{{\mathbb Q}}
\def\R{{\mathbb R}}
\def\Z{{\mathbb Z}}
%--------------------------------------------------------------------------
% Python dans le texte
\def\type#1{\text{\normalfont \ttfamily #1}}
\def\Type#1{\text{\normalfont\bfseries\ttfamily #1}}
%--------------------------------------------------------------------------
% remplace les comparateurs avec égalité, une barre oblique
\let\le=\leqslant 
\let\ge=\geqslant 
%--------------------------------------------------------------------------
% un d droit pour les intégrales et les dérivées.
\def \d{\text{d}}
%--------------------------------------------------------------------------
% Écritures des composantes
%% S'il n'y a pas de solutions
\def\chapitre#1{\input{#1}}
%% S'il y a des solutions
\newif\ifreponses % on crée un indicateur
\reponsesfalse    % par défaut, pas de solutions
\def\chapitreSol#1{
\input{#1}
\ifreponses
     \newpage 
     \section{Solutions} 
     \shipoutAnswer
\fi}
