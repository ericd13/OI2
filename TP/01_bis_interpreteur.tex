%-------------------------------------------------------------------------------
%-------------------------------------------------------------------------------
%-------------------------------------------------------------------------------
%-------------------------------------------------------------------------------
\chapter{I bis -- Formules arithmétiques}
%-------------------------------------------------------------------------------
%-------------------------------------------------------------------------------
\thispagestyle{empty}
%-------------------------------------------------------------------------------
%-------------------------------------------------------------------------------
{\sf 
Dans ce TP nous allons interpréter les programmes dans un langage très simple : les expressions arithmétiques. Le but est de lire une expression arithmétique sous forme d'une chaîne de caractères et d'en calculer la valeur.

Nous nous restreindrons au cas de valeurs entières positives (le cas des entiers négatifs est compliqué par le double sens du symbole "-") et des opérations "+", "-", "*" et "/".}

{\bf Rappels}

\begin{itemize}
    \item Une chaîne de caractères est encadrée par des guillemets doubles : \type{"Bonjour"}.
    \item Sa longueur est fournie par \type{String.length ch}.
    \item Le caractère d'indice $i$ d'une chaîne est obtenu par \type{ch.[i]}.
    \item Un caractère est encadré par des guillemets simples.
    \item \type{Char.code} permet de renvoyer le code ascii d'un caractère.
    \item pour calculer le chiffre représenté par un caractère entre \type{'0'} et \type{'9'}, on peut écrire 
    
    \type{Char.code c - Char.code '0'}. \type{'5'} donnera le chiffre 5.
    \item \type{List.rev} permet de retourner une liste.
\end{itemize}
%-------------------------------------------------------------------------------
%-------------------------------------------------------------------------------
\section{Lexer}
%-------------------------------------------------------------------------------
%-------------------------------------------------------------------------------
La première étape consiste à lire la chaîne de caractères et à la séparer en ses objets de bases appelés lexèmes : ce sera les entiers, les opérateurs et les parenthèses.

On utilisera le type suivant :
\begin{ocaml}
type lexeme =  Nb of int (* les entiers *)
              |Plus |Moins |Fois |Divise (* les opérateurs *)
              |ParO |ParF;; (* les parenthèses *)
\end{ocaml}
%-------------------------------------------------------------------------------
%-------------------------------------------------------------------------------
\begin{question}{}{} 
Écrire une fonction \type{lecture : string -> lexeme list} qui reçoit une chaîne de caractères et qui renvoie la liste des lexèmes associée. Les caractères qui ne correspondent pas à un lexème seront ignorés.
\type{lecture "5*(41 + 3) a"} doit renvoyer
\begin{ocaml}[]
[Nb 5; Fois; ParO; Nb 41; Plus; Nb 3; ParF]
\end{ocaml}

\reponse

\begin{ocaml}
let c0 = Char.code '0';;
let add k liste =
    if k = 0 
    then liste
    else (Nb k) :: liste;;

let lecture texte =
   let n = String.length texte in
   let rec lire i k fait =
      if i = n
      then add k fait
      else match texte.[i] with
           |c when Char.code c >= c0 && Char.code c <= c0 + 9
                -> lire (i+1) (10*k + Char.code c - c0) fait
           |'+' -> lire (i+1) 0 (Plus   :: (add k fait))
           |'-' -> lire (i+1) 0 (Moins  :: (add k fait))
           |'*' -> lire (i+1) 0 (Fois   :: (add k fait))
           |'/' -> lire (i+1) 0 (Divise :: (add k fait))
           |'(' -> lire (i+1) 0 (ParO   :: (add k fait))
           |')' -> lire (i+1) 0 (ParF   :: (add k fait))
           |_   -> lire (i+1) 0 (add k fait) in
   List.rev (lire 0 0 []);;

\end{ocaml}
\end{question}
%-------------------------------------------------------------------------------
%-------------------------------------------------------------------------------
\section{Notation polonaise inversée}
%-------------------------------------------------------------------------------
%-------------------------------------------------------------------------------
En 1920 le mathématicien polonais Jan Lukasiewicz propose la notation pré-fixé (ou polonaise) qui consiste à considérer les opérations comme des opérateurs à 2 variables, $7 + 11$ s'écrit $+\; 7\; 11$. L'avantage est que les parenthèses deviennent inutiles :


 $(7-2).\sin(2x+\pi/3)$ devient
$*\; -\; 7\; 2\; \sin \; +\; *\; 2\; x\; /\; \pi\; 3$.

Elle a été diffusée dans le public comme interface utilisateur avec les calculatrices de bureau de Hewlett-Packard (HP-9100), puis avec la calculatrice scientifique HP-35 en 1972.

Elle consiste simplement à placer l'opérateur {\bf après} les deux opérandes, \type{"7 + 11"} s'écrit \type{"7 11 +"}. L'expression \type{"32 - 2*((7 / 2)*3 - 12)"} peut s'écrire
\type{"32 2 7 2 / 3 * 12 - * -"}

L'avantage est que, combinée à une pile, cette notation permet d'effectuer les calculs sans faire référence à une quelconque adresse mémoire. 

\begin{itemize}
\item On lit l'expression terme-à-terme :

\item si on lit une valeur numérique, elle est empilée, 

\item si on lit une opération $\clubsuit$, 

on dépile les deux derniers opérandes $a$ et $b$ 

et on empile le résultat du calcul $a\ \clubsuit\ b$.
\end{itemize}	
{\bf Un exemple}
On part de l'expression \type{"32 2 7 2 / 3 * 12 - * -"} :

\begin{center}
\begin{tabular}{|c|l|ccccc|}
Lecture&Action&\multicolumn{5}{c}{Pile}\\
\hline
&On crée une pile vide & & & & & \\
\hline
32&On empile & 32 & & & & \\
2&On empile & 32 &2 & & & \\
7&On empile & 32 &2 &7 & & \\
2&On empile & 32 &2 &7 &2 &\ \\
$/$&On dépile 2 éléments& 3 &2 & & &\ \\
&On empile $7/2$& 32 &2 &3 & &\ \\
3&On empile & 32 &2 &3 &3 &\ \\
$*$&On dépile 2 éléments& 32 &2 & & &\ \\
&On empile $3*3$& 32 &2 &9 & &\ \\
12&On empile & 32 &2 &9 &12 &\ \\
-&On dépile 2 éléments& 32 &2 & & &\ \\
&On empile $9-12$& 32 &2 & &-3 &\ \\
$*$&On dépile 2 éléments&32 & & & &\ \\
&On empile $2*(-3)$&32 &-6 & & &\ \\
$-$&On dépile 2 éléments& & & & &\ \\
&On empile $32-(-6)$&38 & & & &\ \\
\hline
&On dépile le résultat : 38& & & & &\ \\
\hline
\end{tabular}
\end{center}

%-------------------------------------------------------------------------------
%-------------------------------------------------------------------------------
\begin{question}{}{}
Définir un type \type{pile} pour une pile {\bf impérative} et écrire les fonctions \type{créerPile}, \type{estrPileVide}, \type{empiler}, \type{voir} et \type{depiler}. \type{depiler} devra enlever l'élément au sommet de la pile {\bf et} le renvoyer en même temps.

\reponse

\begin{ocaml}
type 'a pile = {contenu = 'a list};;

let creerPile () = {contenu = []};;

let estPileVide pile =
   pile.contenu = [];;

let empiler x pile = 
   pile.contenu  <- x :: pile.contenu;;
   
let depiler pile = 
   match pile.contenu with
   |[] -> failwith "La pile est vide"
   |t::q -> pile.contenu <- q;
            t;;
            
let voir pile = List.hd pile.contenu;;
\end{ocaml}
% On pourra aussi utiliser une pile persistante.
% \begin{ocaml}
% let creerPile () = [];;

% let empiler x pile = x::pile;;

% let dépiler pile = 
%   match pile with
%   |[] -> failwith "La pile est vide"
%   |t::q -> t, q;;
% \end{ocaml}
\newpage
\end{question}
%-------------------------------------------------------------------------------
%-------------------------------------------------------------------------------
\begin{question}{}{}
Écrire une fonction \type{calculNPI : string -> int} qui calcule la valeur d'une expression {\bf écrite au format NPI} dans une liste de lexèmes et qui calcule sa valeur.

\reponse

\begin{ocaml}
let depiler2 pile =
    let b = depiler pile in
    let a = depiler pile in
    a, b;;

let calculNPI liste =
   let pile = creerPile () in
   let rec traiter liste =
      match liste with
      |[] -> depiler pile
      |(Nb k)::q -> empiler k pile;
                    traiter q
      |Plus::q -> let a, b = depiler2 pile in 
                  empiler (a+b) pile;
                  traiter q 
      |Moins::q -> let a, b = depiler2 pile in 
                   empiler (a-b) pile;
                   traiter q
      |Fois::q -> let a, b = depiler2 pile in 
                  empiler (a*b) pile;
                  traiter q
      |Divise::q -> let a, b = depiler2 pile in 
                    empiler (a/b) pile;
                    traiter q
       |_::q -> traiter q  in   
   traiter liste;;        
\end{ocaml}
% Si on utilise une pile persistante, il faut alors l'envoyer comme paramètre.
% \begin{ocaml}
% let depiler2 pile =
%     let b, pile1 = depiler pile in
%     let a, pile2 = depiler pile1 in
%     a, b, pile2;;

% let calculerNPI texte =
%   let rec traiter liste pile =
%       match liste with
%       |[] -> fst (depiler pile)
%       |(Nb k)::q -> traiter q (empiler k pile)
%       |Plus::q -> let a, b, pile1 = depiler2 pile in 
%                   traiter q (empiler (a+b) pile1)
%       |Moins::q -> let a, b, pile1 = depiler2 pile in 
%                   traiter q (empiler (a-b) pile1)
%       |Fois::q -> let a, b, pile1 = depiler2 pile in 
%                   traiter q (empiler (a*b) pile1)
%       |Divise::q -> let a, b, pile1 = depiler2 pile in 
%                     traiter q (empiler (a/b) pile1)
%       |_::q -> traiter q pile in   
%   traiter (lecture texte) (creerPile ());;        
% \end{ocaml}
% \newpage
\end{question}
%-------------------------------------------------------------------------------
\newpage
%-------------------------------------------------------------------------------
\section{Cas général}
%-------------------------------------------------------------------------------
%-------------------------------------------------------------------------------
Le procédé ci-dessus demande d'écrire la formule mathématique en notation N.P.I. ce qui n'est pas pratique. Nous allons maintenant traduire une formule "normale" en une formule N.P.I.

Pour cela on aura à gérer la priorité des opérations :

\type{"3 + 4 * 5"} est traduit en \type{"3 4 5 * +"} alors que
\type{"3 * 4 + 5"} est traduit en \type{"3 4 * 5 +"}

On donne la priorité 1 aux opérations $+$ et $-$ et la priorité 2 aux opérations $*$ et $/$.

\medskip

Pour traduire une formule on effectue les opérations suivantes.
\begin{itemize}
\item On lit la formule terme-à-terme :

\item on crée une pile vide :

\item si on lit une valeur numérique, elle est ajoutée à la sortie, 

\item si on lit une opération $\clubsuit$, on dépile les opérations de priorité supérieure ou égale à celle de $\clubsuit$ et on les ajoute à la sortie, on empile ensuite $\clubsuit$,

\item si on lit une parenthèse ouvrante on l'empile, 

\item si on lit une parenthèse fermante, on dépile et on ajoute à la sortie toutes les opérations jusqu'à ce qu'on trouve une parenthèse ouvrante, on dépile cette parenthèse ouvrante.

\item Quand on a lu tous les lexèmes, on dépile et on ajoute à la sortie toutes les opérations resytantes.

\item La sortie est une liste que l'on doit renverser pour obtenir la formule N.P.I.
\end{itemize}	

{\bf Un exemple}
On part de l'expression \type{"2*(11-2+3*2)"} qui a été transcrite en 
\begin{ocaml}[]
[Nb 2; Fois; ParO; Nb 11; Moins; Nb 2; Plus; 
                                 Nb 3; Fois; Nb 2; ParF]
\end{ocaml}

Pour des raisons de place on écrit les lexèmes sous forme "naturelle" dans la pile et la sortie.
\begin{center}
\begin{tabular}{|c|cccc|r|}
Lecture&\multicolumn{4}{c}{Pile}&Sortie\\
\hline
\texttt{Nb 2} &&&&&\texttt{[2]}\\
\texttt{Fois}&\texttt{(*, 2)}&&&&\texttt{[2]}\\
\texttt{ParO}&\texttt{(*, 2)}&\texttt{((, 0)}&&&\texttt{[2]}\\
\texttt{Nb 11}&\texttt{(*, 2)}&\texttt{((, 0)}&&&\texttt{[11; 2]}\\
\texttt{Moins}&\texttt{(*, 2)}&\texttt{((, 0)}&\texttt{(-, 1)}&&\texttt{[11; 2]}\\
\texttt{Nb 2}&\texttt{(*, 2)}&\texttt{((, 0)}&\texttt{(-, 1)}&&\texttt{[2; 11; 2]}\\
\texttt{Plus}&\texttt{(*, 2)}&\texttt{((, 0)}&\texttt{(+, 1)}&&\texttt{[-; 2; 11; 2]}\\
\texttt{Nb 3}&\texttt{(*, 2)}&\texttt{((, 0)}&\texttt{(+, 1)}&&\texttt{[3; -; 2; 11; 2]}\\
\texttt{Fois}&\texttt{(*, 2)}&\texttt{((, 0)}&\texttt{(+, 1)}&\texttt{(*, 2)}&\texttt{[3; -; 2; 11; 2]}\\
\texttt{Nb 2}&\texttt{(*, 2)}&\texttt{((, 0)}&\texttt{(+, 1)}&\texttt{(*, 2)}&\texttt{[2; 3; -; 2; 11; 2]}\\
\texttt{ParF}&\texttt{(*, 2)}&&&&\texttt{[+; *; 2; 3; -; 2; 11; 2]}\\
\hline
&&&&&\texttt{[*; +; *; 2; 3; -; 2; 11; 2]}\\
\hline
\end{tabular}
\end{center}
La notation N.P.I. est donc 
\begin{ocaml}[]
[Nb 2; Nb 11; Nb 2; Moins; Nb 3; Nb 2; Fois; Plus; Fois]
\end{ocaml}
%-------------------------------------------------------------------------------
%-------------------------------------------------------------------------------
\begin{question}{}{}
Écrire une fonction \type{convertirNPI : lexeme list -> lexeme list} qui convertit une liste de lexèmes qui exprime une expression sous forme naturelle en une liste de lexèmes qui exprime une expression sous forme N.P.I.

\reponse

\begin{ocaml}
let sortir prio pile out =
    while not (estPileVide pile) && snd (voir pile) >= prio do
       empiler (fst (depiler pile)) out done;;  
       
let vider pile =
   let rec transfert pile out =
      if estPileVide pile
      then out
      else let x = depiler pile in transfert pile (x::out)
   in transfert pile [];;  
\end{ocaml}
\newpage           
\begin{ocaml}
let convertirNPI liste =
   let ope = creerPile () in
   let out = creerPile () in
   let rec traiter liste =
      match liste with
      |(Nb k)::reste -> empiler (Nb k) out; 
                        traiter reste
      |Plus::reste -> sortir 1 ope out;
                      empiler (Plus, 1) ope;
                      traiter reste
      |Moins::reste -> sortir 1 ope out;
                       empiler (Moins, 1) ope;
                       traiter reste
      |Fois::reste -> sortir 2 ope out;
                      empiler (Fois, 2) ope;
                      traiter reste
      |Divise::reste -> sortir 2 ope out;
                        empiler (Divise, 1) ope;
                        traiter reste
      |ParO::reste -> empiler (ParO, 0) ope; 
                      traiter reste
      |ParF::reste -> sortir 1 ope out;
                      let _ = depiler ope in traiter reste
      |[] -> sortir 1 ope out;
             vider out
   in traiter liste;;
\end{ocaml}
\end{question}
%-------------------------------------------------------------------------------
%-------------------------------------------------------------------------------
\begin{question}{}{}
En déduire une fonction \type{calcul : string -> int} qui calcule la valeur d'une expression naturelle dans une chaîne de caractères et qui calcule sa valeur.

\reponse

\begin{ocaml}
let calcul ch =
   calculNPI (convertirNPI (lecture ch));;
\end{ocaml}
\end{question}
%-------------------------------------------------------------------------------
%-------------------------------------------------------------------------------
