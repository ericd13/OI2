%-------------------------------------------------------------------------------
%-------------------------------------------------------------------------------
%-------------------------------------------------------------------------------
\chapter{Automates}
%-------------------------------------------------------------------------------
%-------------------------------------------------------------------------------
\thispagestyle{empty}
%-------------------------------------------------------------------------------
%-------------------------------------------------------------------------------
%-------------------------------------------------------------------------------
\section{Automates déterministes} 
%-------------------------------------------------------------------------------
%-------------------------------------------------------------------------------
%-------------------------------------------------------------------------------
On choisit de représenter les automates déterministes complets par le type
%-------------------------------------------------------------------------------
\begin{lstlisting}
type 'a afd = {indice : 'a -> int;
               delta  : int array array;
               finaux : bool array};;
\end{lstlisting}
%-------------------------------------------------------------------------------
\begin{itemize}
    \item \type{delta} est une matrice de taille $n\times p$.
    \item \type{finaux} est un tableau de taille $n$.
    \end{itemize}
%-------------------------------------------------------------------------------

\medskip

Un enregistrement représente un automate $Q = ({\cal A}, S, \delta, s_0, F)$ avec
%-------------------------------------------------------------------------------
\begin{itemize}
    \item ${\cal A}$ est défini par l'ensemble de départ de la fonction \type{indice} qui envoie bijectivement l'alphabet dans $\{0, 1, 2, \ldots, p-1\}$.
    \item L'ensemble des états est $S = \{0, 1, 2, n-1\}$.
    \item $\delta(s, x)$ est \type{delta.(s).(indice x)}
    \item $s_0=0$
    \item $F$ est l'ensemble des états $s$ tels que \type{finaux.(s)} vaut \type{false}.
\end{itemize}
%-------------------------------------------------------------------------------

\medskip

{\bf Nous utiliserons principalement des automates où les lettres sont des caractères.}

On pourra utiliser comme exemple l'automate $Q_1$ défini par 
%-------------------------------------------------------------------------------
\begin{lstlisting}
let indice1 x = 
  match x with 
  |'a' -> 0
  |'b' -> 1
  |_ -> failwith "Lettre bon valide";;

let delta1 = [|[|1; 2|]; [|2; 4|]; [|4; 0|]; 
               [|2; 4|]; [|2; 1|]|];;

let q1 = {indice = indice1; delta = delta1; 
          finaux = [|false; false; false; false; true|]};;
\end{lstlisting}
%-------------------------------------------------------------------------------
\begin{figure}
\centering
\begin{tikzpicture}[baseline,
                    accepting=accepting by gdouble, 
                    initial text =,
                    scale = 1.5,
                    bend angle=15]
\tikzset{sommet/.style = {anchor=base,circle, draw, minimum size = 20 pt, >=latex}}                
\node[sommet,initial] (S0) at  (0,0) {0};
\node[sommet] (S1) at  (2,2) {1};
\node[sommet] (S2) at  (2,0) {2};
\node[sommet] (S3) at  (4,0) {3};
\node[sommet,accepting] (S4) at  (4,2) {4};
\draw[->] (S0) edge             node[above left] {$a$} (S1);
\draw[->] (S0) edge[bend left]  node[above]      {$b$} (S2);
\draw[->] (S1) edge             node[left]       {$a$} (S2);
\draw[->] (S1) edge[bend left]  node[above]      {$b$} (S4);
\draw[->] (S2) edge[bend left]  node[below]      {$b$} (S0);
\draw[->] (S2) edge[bend left]  node[above left] {$a$} (S4);
\draw[->] (S3) edge             node[below]      {$a$} (S2);
\draw[->] (S3) edge             node[right]      {$b$} (S4);
\draw[->] (S4) edge[bend left]  node[below right]{$a$} (S2);
\draw[->] (S4) edge[bend left]  node[below]      {$b$} (S1);
\end{tikzpicture}
\caption{L'automate $Q_1$}
\end{figure}
%-------------------------------------------------------------------------------
\newpage
%-------------------------------------------------------------------------------
\begin{Exercise}\it 
Écrire une fonction qui reçoit une chaîne de caractère et un automate et qui renvoie \type{true} ou \type{false} selon que la chaîne définit un mot reconnu par l'automate on non.
\end{Exercise} 
%--------------------------------------------------------------------------
\begin{Answer}
\begin{lstlisting}
let reconnu mot aut =
  let ind = aut.indice in
  let d = aut.delta in
  let f = aut.finaux in
  let s = ref 0 in
  let n = String.length mot in
  for i = 0 to (n-1) do
    s := d.(!s).(ind mot.[i]) done;
  f.(!s);;
\end{lstlisting}
\end{Answer} 
%-------------------------------------------------------------------------------
\begin{lstlisting}
reconnu : string -> char afd -> bool
\end{lstlisting}
%-------------------------------------------------------------------------------
%-------------------------------------------------------------------------------
La longueur d'une chaîne est calculée par \type{String.length}.

Le $k$-ième caractère est accessible par \type{ch.[k]} pour $0\le k < $\type{String.length ch}.
%-------------------------------------------------------------------------------
%-------------------------------------------------------------------------------
\begin{Exercise}\it 
Écrire une fonction qui calcule les états accessibles d'un automate sous la forme d'un tableau de booléens (de longueur $n$). Un état $s$ sera accessible si et seulement si la tableau vaut \type{true} en $s$.
\end{Exercise} 
%--------------------------------------------------------------------------
\begin{Answer}
\begin{lstlisting}
let accessibles aut =
  let d = aut.delta in
  let n = Array.length d in
  let p = Array.length d.(0) in
  let acc = Array.make n false in
  let rec aux s =
    if not acc.(s)
    then begin acc.(s) <- true;
    for i = 0 to (p-1) do 
      aux d.(s).(i) done end in
  aux 0;
  acc;;
\end{lstlisting}
\end{Answer} 
%-------------------------------------------------------------------------------
\begin{lstlisting}
accessibles : 'a afd -> bool array
\end{lstlisting}
%-------------------------------------------------------------------------------
%-------------------------------------------------------------------------------
\begin{Exercise}\it 
En déduire une fonction qui émonde un automate.
\end{Exercise} 
%--------------------------------------------------------------------------
\begin{Answer}
\begin{lstlisting}
let emonder aut =
  let d = aut.delta in
  let n = Array.length d in
  let p = Array.length d.(0) in
  let acc  = accessibles aut in
  let a = Array.make n 0 in 
  let b = Array.make n 0 in 
  let q = ref 0 in
  for i = 0 to (n-1) do
    if acc.(i) 
    then begin a.(i) <- !q; b.(!q) <- i; 
              incr q end done;
  let new_delta = Array.make !q [||] in
  let new_finaux = Array.make !q false in
  for i = 0 to (!q -1) do
    new_delta.(i) <- Array.make p 0;
    for j = 0 to (p-1) do 
      new_delta.(i).(j) <- a.(d.(b.(i)).(j)) done;
    new_finaux.(i) <- aut.finaux.(b.(i)) done;
  {indice = aut.indice; delta = new_delta; 
   finaux = new_finaux};;
\end{lstlisting}
\newpage
\end{Answer} 
%-------------------------------------------------------------------------------
\begin{lstlisting}
emonder : 'a afd -> 'a afd
\end{lstlisting}
%-------------------------------------------------------------------------------
%-------------------------------------------------------------------------------
%-------------------------------------------------------------------------------
\section{Automates non déterministes} 
%-------------------------------------------------------------------------------
%-------------------------------------------------------------------------------
%-------------------------------------------------------------------------------
Les automates non déterministes sont représentés par la type
%-------------------------------------------------------------------------------
\begin{lstlisting}
type 'a afnd = {n_indice : 'a -> int;
                n_delta : int list array array;
                n_initiaux : int list;
                n_finaux : int list};;
\end{lstlisting}
%-------------------------------------------------------------------------------
\begin{itemize}
    \item \type{n\_delta} est une matrice de listes de taille $n\times p$.
    \item \type{n\_initiaux} et \type{n\_finaux} sont des tableaux de taille $n$.
    \item On supposera que les listes d'entiers sont {\bf triées par ordre croissant et sans doublon}.
    \end{itemize}
%-------------------------------------------------------------------------------

\medskip

Un enregistrement représente un automate $Q = ({\cal A}, S, \Delta, I, F)$ avec
%-------------------------------------------------------------------------------
\begin{itemize}
    \item ${\cal A}$ est défini par l'ensemble de départ de la fonction \type{n\_indice} qui envoie bijectivement l'alphabet dans $\{0, 1, 2, \ldots, p-1\}$.
    \item L'ensemble des états est $S = \{0, 1, 2, n-1\}$.
    \item $\Delta(s, x)$ est l'ensemble représenté par la liste \type{n\_delta.(s).(indice x)}.
    \item $I$ est l'ensemble représenté par la liste \type{n\_initiaux}.
    \item $F$ est l'ensemble représenté par la liste \type{n\_finaux}.
\end{itemize}
%-------------------------------------------------------------------------------

\newpage

On pourra utiliser comme exemple l'automate $Q_2$ défini par 
%-------------------------------------------------------------------------------
\begin{lstlisting}
let delta2 = [|[|[0; 1]; [0]|]; [|[2]; [0]|]; [|[3]; [0]|]; 
               [|[3]; [4]|]; [|[]; []|];|];;

let q2 = {n_indice = indice1;
          n_delta = delta2;
          n_initiaux = [0];
          n_finaux = [4]};;
\end{lstlisting}
%-------------------------------------------------------------------------------
\begin{figure}
\centering
\begin{tikzpicture}[baseline,
                    accepting=accepting by gdouble, 
                    initial text =,
                    scale = 1.2,
                    bend angle=15]
\tikzset{sommet/.style = {anchor=base,circle, draw, minimum size = 20 pt, >=latex}}                
\node[sommet,initial] (S0) at  (0,0) {0};
\node[sommet] (S1) at  (2,0) {1};
\node[sommet] (S2) at  (4,0) {2};
\node[sommet] (S3) at  (6,0) {3};
\node[sommet,accepting] (S4) at  (8, 0) {4};
\draw[->] (S0) edge[loop above] node[above]      {$a$} (S0);
\draw[->] (S0) edge[loop below] node[below]      {$b$} (S0);
\draw[->] (S0) edge             node[above]      {$a$} (S1);
\draw[->] (S1) edge             node[above]      {$a$} (S2);
\draw[->] (S2) edge             node[above]      {$a$} (S3);
\draw[->] (S3) edge             node[above]      {$b$} (S4);
\end{tikzpicture}
\caption{L'automate $Q_2$}
\end{figure}
%-------------------------------------------------------------------------------
%-------------------------------------------------------------------------------
\begin{Exercise}\it 
Écrire une fonction qui fusionne deux listes triées par ordre croissant et sans doublon en une liste avec les mêmes caractéristiques.
\end{Exercise} 
%--------------------------------------------------------------------------
\begin{Answer}
\begin{lstlisting}
let rec fusion l1 l2 = 
  match l1, l2 with
  |[], _ -> l2
  |_, [] -> l1
  |t1::q1, t2::q2 when t1 = t2 -> t1 :: (fusion q1 q2)
  |t1::q1, t2::q2 when t1 < t2 -> t1 :: (fusion q1 l2)
  |t1::q1, t2::q2 -> t2 :: (fusion l1 q2);;
\end{lstlisting}
\end{Answer} 
%-------------------------------------------------------------------------------
\begin{lstlisting}
fusion : 'a list -> 'a list -> 'a list
\end{lstlisting}
%-------------------------------------------------------------------------------
%-------------------------------------------------------------------------------
\begin{Exercise}\it 
Écrire une fonction qui détermine si deux listes triées par ordre croissant et sans doublon ont une valeur commune.
\end{Exercise} 
%--------------------------------------------------------------------------
\begin{Answer}
\begin{lstlisting}
let rec inter l1 l2 =
  match l1, l2 with
  |[], _ -> false
  |_, [] -> false
  |t1::q1, t2::q2 when t1 = t2 -> true
  |t1::q1, t2::q2 when t1 < t2 -> inter q1 l2
  |t1::q1, t2::q2 -> inter l1 q2;;
\end{lstlisting}
\end{Answer} 
%-------------------------------------------------------------------------------
\begin{lstlisting}
inter : 'a list -> 'a list -> bool 
\end{lstlisting}
%-------------------------------------------------------------------------------
%-------------------------------------------------------------------------------
\begin{Exercise}\it 
En déduire une fonction qui reçoit une chaîne de caractère et un automate non déterministe et qui renvoie \type{true} ou \type{false} selon que la chaîne définit un mot reconnu par l'automate on non.
\end{Exercise} 
%--------------------------------------------------------------------------
\begin{Answer} On commence par une fonction qui calcule l'image d'un ensemble.
\begin{lstlisting}
let rec delta_liste delta liste k =
  match liste with
  |[] -> []
  |t::q -> fusion delta.(t).(k) (delta_liste delta q k);;
\end{lstlisting}

\begin{lstlisting}
let n_reconnu mot aut =
  let ind = aut.n_indice in
  let d = aut.n_delta in
  let f = aut.n_finaux in
  let s = ref aut.n_initiaux in
  let n = String.length mot in
  for i = 0 to (n-1) do
    s := delta_liste d !s (ind mot.[i]) done;
  inter !s f;;
\end{lstlisting}
\end{Answer} 
%-------------------------------------------------------------------------------
\begin{lstlisting}
n_reconnu : string -> char afnd -> bool
\end{lstlisting}
%-------------------------------------------------------------------------------
%-------------------------------------------------------------------------------
\begin{Exercise}\it 
Écrire une fonction 
\end{Exercise} 
%--------------------------------------------------------------------------
\begin{Answer}
\begin{lstlisting}
\end{lstlisting}
\end{Answer} 
%-------------------------------------------------------------------------------
\begin{lstlisting}
reconnu : string -> char afd -> bool
\end{lstlisting}
%-------------------------------------------------------------------------------
%-------------------------------------------------------------------------------
%-------------------------------------------------------------------------------
%-------------------------------------------------------------------------------
\begin{Exercise}\it 
Écrire une fonction 
\end{Exercise} 
%--------------------------------------------------------------------------
\begin{Answer}
\begin{lstlisting}
\end{lstlisting}
\end{Answer} 
%-------------------------------------------------------------------------------
\begin{lstlisting}
reconnu : string -> char afd -> bool
\end{lstlisting}
%-------------------------------------------------------------------------------
%-------------------------------------------------------------------------------
La longueur d'une chaîne est calculée par \type{String.length}.

Le $k$-ième caractère est accessible par \type{ch.[k]} pour $0\le k < $\type{String.length ch}.
%-------------------------------------------------------------------------------
%-------------------------------------------------------------------------------
\begin{Exercise}\it 
Écrire une fonction qui calcule les états accessibles d'un automate sous la forme d'un tableau de booléens (de longueur $n$). Un état $s$ sera accessible si et seulement si la tableau vaut \type{true} en $s$.
\end{Exercise} 
%--------------------------------------------------------------------------
\begin{Answer}
\begin{lstlisting}
let accessibles aut =
  let d = aut.delta in
  let n = Array.length d in
  let p = Array.length d.(0) in
  let acc = Array.make n false in
  let rec aux s =
    if not acc.(s)
    then begin acc.(s) <- true;
    for i = 0 to (p-1) do 
      aux d.(s).(i) done end in
  aux 0;
  acc;;
\end{lstlisting}
\end{Answer} 
%-------------------------------------------------------------------------------
\begin{lstlisting}
accessibles : 'a afd -> bool array
\end{lstlisting}
%-------------------------------------------------------------------------------
%-------------------------------------------------------------------------------
\begin{Exercise}\it 
En déduire une fonction qui émonde un automate.
\end{Exercise} 
%--------------------------------------------------------------------------
\begin{Answer}
\begin{lstlisting}
let emonder aut =
  let d = aut.delta in
  let n = Array.length d in
  let p = Array.length d.(0) in
  let acc  = accessibles aut in
  let a = Array.make n 0 in 
  let b = Array.make n 0 in 
  let q = ref 0 in
  for i = 0 to (n-1) do
    if acc.(i) 
    then begin a.(i) <- !q; b.(!q) <- i; 
              incr q end done;
  let new_delta = Array.make !q [||] in
  let new_finaux = Array.make !q false in
  for i = 0 to (!q -1) do
    new_delta.(i) <- Array.make p 0;
    for j = 0 to (p-1) do 
      new_delta.(i).(j) <- a.(d.(b.(i)).(j)) done;
    new_finaux.(i) <- aut.finaux.(b.(i)) done;
  {indice = aut.indice; delta = new_delta; 
   finaux = new_finaux};;
\end{lstlisting}
\newpage
\end{Answer} 
%-------------------------------------------------------------------------------
\begin{lstlisting}
emonder : 'a afd -> 'a afd
\end{lstlisting}
%-------------------------------------------------------------------------------
%-------------------------------------------------------------------------------






