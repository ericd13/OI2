%-------------------------------------------------------------------------------
%-------------------------------------------------------------------------------
%-------------------------------------------------------------------------------
\chapter{DS3 CCINP 2020}
%-------------------------------------------------------------------------------
%-------------------------------------------------------------------------------
\setcounter{Exercise}9
Les questions posées sont beaucoup plus facilement résolues en Python grâce à la structure hybride des listes python : on les construit par adjonction, comme les listes OCaml mais on les lit séquentiellement, comme les tableaux Caml.
%-------------------------------------------------------------------------------
%-------------------------------------------------------------------------------
%-------------------------------------------------------------------------------
\begin{Exercise}
Pierre peut connaître immédiatement $x$ et et $y$ si et seulement si $\Pi$ est le produit de deux nombres premiers distincts.
\end{Exercise}
%-------------------------------------------------------------------------------
%-------------------------------------------------------------------------------
On a la majoration $x+y = S\le n$. On en déduit $P = x.y \le x.(n-x)=\frac{n^2}4 - \bigl(x-\frac n2\bigr)^2\le \frac{n^2}4$. Quand on cherchera un nombre de produits on pourra classer les produits comme les indices d'un tableau de taille $\left\lfloor\frac{n^2}4\right\rfloor+1$.
%-------------------------------------------------------------------------------
%-------------------------------------------------------------------------------
\begin{Exercise}
% Si on a $2\le x <y$  et $x+y \le n$ alors $y\le n-2$ et $x\le n-3$ d'où $x.y< n^2$.
% \begin{lstlisting}
% let coupleProd n = 
%   let prods = Array.make (n*n) 0 in
%   for x = 2 to (n-1) do
%       for y = (x+1) to (n - x) do
%           let p = x*y in prods.(p) <- prods.(p) +1 done done;
%   let listeP = ref [] in
%   for p = 0 to (n*n-1) do
%       if prods.(p) > 1 then listeP := p :: !listeP done;
%   List.rev !listeP;;
% \end{lstlisting}
% On retourne la liste pour qu'elle soit dans l'ordre croissant.
\begin{lstlisting}[language=python]
def CoupleProd(n):
    m = (n**2)//4+1
    nb_prods = [0]*m
    for x in range(2, n):
        for y in range(x+1, n-x+1):
            p = x*y
            nb_prods[p]  += 1
    listeP = []
    for p in range(m):
        if nb_prods[p] > 1:
            listeP.append(p)
    return listeP
\end{lstlisting}

On peut simplifier les 5 dernières lignes en 
\begin{lstlisting}[language=python]
    return [p for p in range(m) if nb_prods[p] > 1]
\end{lstlisting}
\end{Exercise}
%-------------------------------------------------------------------------------
%-------------------------------------------------------------------------------
\begin{Exercise}
On cherche les produits de la forme $x.(S-x)$ pour $x\ge 2$ et $x< S-x$ donc $x<\frac S2$.

Or $x \mapsto x.(S-x)$ est strictement croissante sur $[-\infty ;\frac S2[$ donc les valeurs sont distinctes : on n'a pas besoin de tester les doublons. 

Je n'ai pas compris l'utilité de la variable $n$, on suppose $S \le n$ donc $n$ ne sert pas.

\newpage
% \begin{lstlisting}
% let prod s n =
%   let rec aux x = 
%       if 2*x >= s
%       then []
%       else x*(s-x) :: aux (x+1)
%   in aux 2;;
% \end{lstlisting}

\begin{lstlisting}[language=python]
def Prod(S, n):
    prods = []
    x = 2
    while 2*x < S:
        prods.append(x*(S-x))
        x = x + 1
    return prods
\end{lstlisting}
\end{Exercise}
%-------------------------------------------------------------------------------
%-------------------------------------------------------------------------------
\begin{Exercise}
La valeur minimale d'une somme admissible est $S = 2 + 3$.
% \begin{lstlisting}
% let rec inclus l1 l 2 =
%   match l1, l2 with
%   |[], _ -> true
%   |_, [] -> false
%   |t1::q1, t2::q2 when t1 < t2 -> false
%   |t1::q1, t2::q2 when t1 = t2 -> inclus q1 q2
%   |t1::q1, t2::q2 -> inclus l1 q2;;
   
% let candidat_S n = 
%   let listeP = coupleProd n in
%   let rec aux s =
%       if s > n 
%       then []
%       else begin
%          if inclus (prod s n) listeP
%          then s :: (aux (s+1))
%          else aux (s+1) end
%   in aux 5;;
% \end{lstlisting}

% On prend moins de précaution de complexité en Python
\begin{lstlisting}[language=python]
def inclus(liste1, liste2):
    for x in liste1:
        if not x in liste2:
            return False
    return True
    
def Candidat_S(n):
    listeP = CoupleProd(n)
    cand = []
    for S in range(5, n+1):
        if inclus(Prod(S, n), listeP):
            cand.append(S)
    return cand        
\end{lstlisting}
\end{Exercise}
%-------------------------------------------------------------------------------
%-------------------------------------------------------------------------------
\begin{Exercise}

Je ne comprends pas bien l'énoncé, il semble en fait parler de Sophie plutôt que Pierre. Pour que Pierre puisse connaître $\Sigma$, il suffit que, parmi les sommes possibles associées au produit $\Pi$, il y en ait une seule appartenant à \type{Candidat\_S(n)}. La question est utile en fait pour Sophie, même s'il serait plus simple d'écrire une fonction \type{Unique\_P}. 

% \begin{lstlisting}
% let double_P n = 
% let m = (n*n)/4 in
%   let nb_prod = Array.make m 0 in 
%   let compter1 = List.iter (fun k -> nb_prod.(k) <- nb_prod.(k) + 1) in
%   List.iter (fun s -> compter1 (prod s n)) (candidat_s n);
%   let out = ref [] in
%   for i = 6 to (m-1) do
%       if nb_prod.(i) <1 then out := i :: !out done;
%   out;;
% \end{lstlisting}

On reprend le principe de la question \type{CoupleProd}.
\begin{lstlisting}[language=python]
def Double_p(n):
    m = (n*n)//4+1
    nb_prods = [0]*m
    for S in Candidat_S(n):
        for P in Prod(s, n):
            nb_prods[P] += 1
    return  [k for k in range(m) if np_prods[k] > 1]
\end{lstlisting}
\end{Exercise}
%-------------------------------------------------------------------------------
%-------------------------------------------------------------------------------
\begin{Exercise}
On écrit ce qui est demandé bien que cela semble inutile

On commence par calculer l'intersection.

\begin{lstlisting}[language=python]
def inter(liste1, liste2):
    int = []
    for x in liste1:
        if x in liste2:
            int.append(x)
    return int
\end{lstlisting}

\newpage

\begin{lstlisting}[language=python]
def Reste_S(n):
    reste = []
    db = Double_P(n)
    for S in Candidat_S(n):
        int = inter(d, Prod(S, n))
        if len(int) == 1:
            reste.append(S)
    return reste
\end{lstlisting}
\end{Exercise}
%-------------------------------------------------------------------------------
%-------------------------------------------------------------------------------
\begin{Exercise}
\begin{lstlisting}[language=python]
def Reste_P(S, n):
    db = Double_P(n)
    reste = []
    for P in Prod(S, n):
        if not(P in db):
            reste.append(P)
    return reste    
\end{lstlisting}
\end{Exercise}
%-------------------------------------------------------------------------------
%-------------------------------------------------------------------------------
\begin{Exercise}
Il s'agit de la résolution d'une équation de degré 2 ; à quoi sert $n$ ?

% \begin{lstlisting}
% let solution p s n =
%   let p1 = float_of_int p in
%   let s1 = float_of_int s in
%   let d = (s1*.s1 -. 4.0*.p1)**0.5 in
%   (s1 +. d)/.2.0, (s1 -. d)/.2.0;;
% \end{lstlisting}

\begin{lstlisting}[language=python]
def Solution(P, S, n):
    d = (s**2 - 4*p)**0.5
    return (s-d)/2, (s+d)/2
\end{lstlisting}
\end{Exercise}
%-------------------------------------------------------------------------------
%-------------------------------------------------------------------------------
